\usepackage[utf8]{inputenc}

\title{Formalisms\\
Every Computer Scientist Should Know}
\date{}
% \newcommand{\booklicense}{
%     This work is marked with CC0 1.0. 
%     To view a copy of this license, visit http://creativecommons.org/publicdomain/zero/1.0
% }

% Author subtitle
% For example, university or geographical location
% \newcommand{\authorsubtitle}{ISTA}


\synctex=1

% PACKAGES

% this package allows large \fontsize
\usepackage{fix-cm}  
% this is for graphics. e.g. rectangle on title page
\usepackage{tikz}    
% Used by equations
\usepackage{amsmath} 
\usepackage{amssymb,amsfonts,amsthm,mathtools}
\usepackage[shortlabels]{enumitem}

% Paper dimensions
% \usepackage[a4paper, total={6in, 8in}]{geometry}
% Inserting images
\usepackage{graphicx}
% Index creation
\usepackage{makeidx}
% Initialize an index so we can add entries with \index
\makeindex 
% Others
\usepackage{pbox}
\usepackage{stmaryrd}
% Links
\usepackage{hyperref}
% Convenient references
\usepackage{cleveref}
% Derivations written using horizontal lines
\usepackage{mathpartir}
% TikZ figures
\usepackage{tikz}
% algorithms
\usepackage{algorithm}
\usepackage{algpseudocode}
% proof systems
\usepackage{bussproofs}

\usetikzlibrary {petri}
\usetikzlibrary{arrows.meta}
\usetikzlibrary{shapes,snakes}
\usetikzlibrary{fit}


\theoremstyle{plain}
\newtheorem{theorem}{Theorem}[section]
\newtheorem{lemma}[theorem]{Lemma}
\newtheorem{corollary}[theorem]{Corollary}

\theoremstyle{definition}
\newtheorem{definition}{Definition}[section]
\newtheorem{exercise}{Exercise}[section]
\newtheorem{example}{Example}[section]

\theoremstyle{remark}
\newtheorem{remark}{Remark}[section]
\newtheorem{metatheorem}{Metatheorem}[section]
\newtheorem{axiom}{Axiom}[section]

\DeclareMathOperator{\lub}{lub}
\DeclareMathOperator{\glb}{glb}
