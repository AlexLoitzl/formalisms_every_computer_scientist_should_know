% \documentclass{article}
% \usepackage{graphicx} % Required for inserting images
% \usepackage{amssymb,amsmath,amsthm}
% \usepackage{stmaryrd}
% \usepackage{listings}
% \usepackage{mathpartir}
% \usepackage[a4paper, portrait]{geometry}

% \newtheorem{theorem}{Theorem}
% \newtheorem{lemma}{Lemma}
% \newtheorem{definition}{Definition}
% \newtheorem{example}{Example}
% \newtheorem{exercise}{Homework}


% \newcommand{\NQ}{\mathbb{Q}}
% \newcommand{\BN}{\mathbb{B}}
\section{IMP}

\begin{remark}
      IMP is a simple imperative language corresponding to PCF.
\end{remark}

\begin{definition}[IMP Syntax]
There are three types of \emph{expressions} in IMP, i.e., 
arithmetic, boolean, and command expressions. Programs in IMP belong to the following grammar:
\begin{align*}
    a &\Coloneqq n \;|\; x \;|\; a + a \;|\; a - a \;|\; a * a \;|\; S a \;|\; 0\\
    b &\Coloneqq \true \;|\; \false  \;|\; a \leq a \;|\; \neg b \;|\; b \land b\\
    c &\Coloneqq \sskip \;|\; x = a \;|\; c; c \;|\; \sif{b}{c}{c}\;|\; \swhile{b}{c}
\end{align*}
\end{definition}

\begin{definition}[IMP Semantics]
    There are three different types of semantics for the IMP programming language. All consider the set of variables $X$ as locations, the define a state as the function $\sigma\colon X \to \ZN$. We use $\bot$ to denote the \emph{undefined} state (e.g., for a non-terminating program).
    Moreover, we denote the set of all states (including $\bot$) as $\Sigma$ ($\Sigma_{\bot}$).
\end{definition}



% \begin{lstlisting}[basicstyle=\ttfamily,mathescape=true]
%     a \coloneqq    n | x 
%        | a $+$ a | a $-$ a | a $*$ a 
%        | S a | 0
%     b \coloneqq    true | false 
%        | a $\leq$ a | $\neg$ b | b $\land$ b
%     c \coloneqq    skip | x = a 
%        | c; c | if b then c else c | while b do c
% % \end{lstlisting}
% where $\imp{x} \in X$ denotes 
% a \emph{location} in our program, 
% among the set of locations $X$.
% We define a \emph{state} of the program
% as a function $\sigma: X \to \ZN$
% and use $\Sigma$ to denote the set of all states.
% We use $\bot$ to denote the \emph{undefined} state 
% (e.g., for a non-terminating program).

\subsection{Operational Semantics}

 \begin{definition}[Operational Semantics]
          The operational semantics for IMP was originally introduced as structured operational semantics (SOS) by Plotkin.
          It is defined by a set of rules.
          We use
        $\langle c, \sigma \rangle \downarrow \sigma' \in \Sigma_\bot$ 
        to denote that, starting with state $\sigma$, 
        a command expression $c$ takes us to state $\sigma'$.
        $\langle a, \sigma \rangle \downarrow n \in \ZN$
        and $\langle b, \sigma \rangle \downarrow v \in \BN$ 
        are defined similarly for arithmetic and Boolean expressions.
        The judgements of the operational semantics are as follows:
        \begin{itemize}
            \item[(i)] for \emph{arithmetic expressions}, note that the same structure can be used for judgements on subtraction and multiplication.
        \end{itemize}
        \begin{center}
            \AxiomC{}
            \UnaryInfC{$\sop{n}{\sigma} \downarrow n$}
            \DisplayProof
            $\quad$
            \AxiomC{}
            \UnaryInfC{$\sop{x}{\sigma} \downarrow \sigma(x)$}
            \DisplayProof
            $\quad$
             \AxiomC{$\sop{a_1}{\sigma} \downarrow n_1 $}
             \AxiomC{$\sop{a_2}{\sigma} \downarrow n_2$}
            \BinaryInfC{$\sop {a_1 + a_2}{\sigma} \downarrow n_1 + n_2$}
            \DisplayProof
        \end{center}
        \item[(ii)] for \emph{boolean expressions}, note that the same structure can be used for completing the set of judgements on Boolean operators.
        \begin{center}
            \AxiomC{}
            \UnaryInfC{$\sop{true}{\sigma} \downarrow \top$}
            \DisplayProof
             $\quad$
             \AxiomC{$ \sop{b_1}{\sigma} \downarrow \top $}
              \AxiomC{$\sop{b_2}{\sigma} \downarrow \top$}
            \BinaryInfC{$\sop{b_1 \land b_2}{\sigma} \downarrow \top$}
            \DisplayProof
        \end{center}
        \begin{center}
            \AxiomC{$\sop{a_1}{\sigma} \downarrow n_1 $}
             \AxiomC{$\sop{a_2}{\sigma} \downarrow n_2$}
              \AxiomC{$n_1 \leq n_2$}
            \TrinaryInfC{$\sop{a_1 \leq a_2}{\sigma} \downarrow \top$}
            \DisplayProof
        \end{center}
        \item[(iii)] for \emph{command expressions}.
         In the expression below, note that $\bot$ in $\sop{b}{\sigma} \downarrow \bot$ is the Boolean false. The other case for $\sif{b}{c_1}{c_2}$ is relatively easy to work out.
         \begin{center}
            \AxiomC{}
            \UnaryInfC{$\sop{skip}{\sigma} \downarrow \sigma$}
            \DisplayProof
             $\quad$
            \AxiomC{$\sop{a}{\sigma} \downarrow n$}
            \UnaryInfC{$\sop{x = a}{\sigma} \downarrow \sigma[x \mapsto n]$}
            \DisplayProof
             $\quad$
            \AxiomC{$\sop{b}{\sigma} \downarrow \bot$}
            \UnaryInfC{$\sop{\swhile{b}{c}}{\sigma} \downarrow \sigma$}
            \DisplayProof
         \end{center}
       \begin{center}
            \AxiomC{$ \sop{b}{\sigma} \downarrow \top $}
             \AxiomC{$ \sop{c}{\sigma} \downarrow \sigma' $}
              \AxiomC{$ \sop{\swhile{b}{c}}{\sigma'} \downarrow \sigma''$}
            \TrinaryInfC{$\sop{\swhile{b}{c}}{\sigma} \downarrow \sigma''$}
            \DisplayProof
        \end{center}
        \begin{center}
            \AxiomC{$\sop{c_1}{\sigma} \downarrow \sigma'$}
            \AxiomC{$\sop{c_2}{\sigma'} \downarrow \sigma''$}
            \BinaryInfC{$\sop{c_1; c_2}{\sigma} \downarrow \sigma''$}
            \DisplayProof
            $\quad$
            \AxiomC{$\sop{b}{\sigma} \downarrow \top$}
            \AxiomC{$\sop{c_1}{\sigma} \downarrow \sigma'$}
            \BinaryInfC{$\sop{\sif{b}{c_1}{c_2}}{\sigma} 
            \rangle \downarrow \sigma'$}
            \DisplayProof
         \end{center}
 \end{definition}

\begin{remark}
  What  we have defined above is 
\emph{multi-step} operational semantics.
We can also define it as rules over \emph{single} steps;
however, there are many versions of single-step semantics, 
depending on the \emph{step size} we pick.  
For single-step semantics, we use
$\sop{c}{\sigma} \to \opsem{c'}{\sigma'}$
to denote that, starting with state $\sigma$, 
a command expression $\imp{c}$ takes us to state $\sigma'$
\emph{after one step}, 
and program $\imp{c'}$ is yet to be executed.
\end{remark}


\begin{example}
    An example of rules in are single-step semantics are as follows.
    \begin{center}
        \AxiomC{$\sop{a}{\sigma} \downarrow n$}
        \UnaryInfC{$\sop{x \coloneqq   a}{\sigma} \to \sop{\sskip}{\sigma[x \mapsto a]}$}
    \DisplayProof
    $\quad$
    \AxiomC{$ \sop{c_1}{\sigma} \to \opsem{skip}{\sigma'}$}
    \UnaryInfC{$\sop{c_1; c_2}{\sigma}
     \to \sop{c_2}{\sigma'}$}
    \DisplayProof
    \end{center}
            \begin{center}
            \AxiomC{$\sop{c_1}{\sigma} \to \sop{c_1'}{\sigma'} $}
            \AxiomC{$\simp{c_1'} \neq \simp{\sskip}$}
            \BinaryInfC{$\sop{c_1; c_2}{\sigma}
     \to \sop{c_1'; c_2}{\sigma'}$}
            \DisplayProof
         \end{center}
\end{example}



\begin{example}
    For the program $\simp{c = x \coloneqq    0; x \coloneqq    1}$, we have:
    \begin{align*}
      \sop{x \coloneqq    0}{\sigma} 
      & \to \sop{\sskip}{\sigma[x \mapsto 0]} \\
      \sop{x \coloneqq    0; x \coloneqq    1}{\sigma} 
      & \to \sop{x \coloneqq    1}{\sigma[x \mapsto 0]} \\
      & \to \sop{skip}{\sigma[x \mapsto 1]}
    \end{align*}
\end{example}




% We have the following set of judgements 
% for operational semantics:

% \begin{enumerate}
    % \item Arithmetic expressions:
    % \[ \infer
    %     { }{\opsem{n}{\sigma} \downarrow n} 
    % \qquad \infer
    %     { }{\opsem{x}{\sigma} \downarrow \sigma(x)}
    % \qquad \infer {
    %     \opsem{a$_1$}{\sigma} \downarrow n_1 \\
    %     \opsem{a$_2$}{\sigma} \downarrow n_2}
    %     {\opsem {a$_1$ + a$_2$}{\sigma} \downarrow n_1 + n_2}
    % \]
    % The same structure can be used 
    % for judgements on subtraction and multiplication.
    % \item Boolean expressions:
    % \[ \infer 
    %     { }{\opsem{true}{\sigma} \downarrow \top}
    % \qquad \infer {
    %     \opsem{a$_1$}{\sigma} \downarrow n_1 \\
    %     \opsem{a$_2$}{\sigma} \downarrow n_2 \\
    %     n_1 \leq n_2}
    % {\opsem{a$_1$ $\leq$ a$_2$}{\sigma} \downarrow \top} \] 
    % \[ \infer {
    %     \opsem{b$_1$}{\sigma} \downarrow \top \\
    %     \opsem{b$_2$}{\sigma} \downarrow \top}
    %     {\opsem{b$_1$ $\land$ b$_2$}{\sigma} \downarrow \top} \]
    % The same structure can be used 
    % for completing the set of judgements on Boolean operators.
    % \item Command expressions:
    % \[ \infer
    %     { }{\opsem{skip}{\sigma} \downarrow \sigma}
    %   \qquad \infer 
    %     {\opsem{a}{,~\sigma} \downarrow n}
    %     {\opsem{x = a}{\sigma} \downarrow \sigma[x \mapsto n]} \]
%     \[ \infer {
%         \opsem{c$_1$}{,~\sigma} \downarrow \sigma' \\
%         \opsem{c$_2$}{,~\sigma'} \downarrow \sigma''}
%         {\opsem{c$_1$; c$_2$}{\sigma} \downarrow \sigma''} \]
%     \[ \infer {
%         \opsem{b}{\sigma} \downarrow \top \\
%         \opsem{c$_1$}{\sigma} \downarrow \sigma'}
%         {\opsem{if b then c$_1$ else c$_2$}{\sigma} 
%             \rangle \downarrow \sigma'} \]
%     \[ \infer {
%         \opsem{b}{\sigma} \downarrow \top \\
%         \opsem{c}{\sigma} \downarrow \sigma' \\
%         \opsem{while b do c}{\sigma'} \downarrow \sigma''}
%         {\opsem{while b do c}{\sigma} \downarrow \sigma''}
%     \qquad \infer 
%         {\opsem{b}{\sigma} \downarrow \bot}
%         {\opsem{while b do c}{\sigma} \downarrow \sigma}
%     \]
%     Note that $\bot$ 
%     in $\opsem{b}{\sigma} \downarrow \bot$
%     is the Boolean false.
%     The other case 
%     for \texttt{if b then c$_1$ else c$_2$}
%     is relatively easy to work out.
% \end{enumerate}

% What  we have defined above is 
% \emph{multi-step} operational semantics.
% We can also define it as rules over \emph{single} steps;
% however, there are many versions of single-step semantics, 
% depending on the \emph{step size} we pick.

% For single-step semantics, we use
% $\opsem{c}{\sigma} \to \opsem{c'}{\sigma'}$
% to denote that, starting with state $\sigma$, 
% a command expression $\imp{c}$ takes us to state $\sigma'$
% \emph{after one step}, 
% and program $\imp{c'}$ is yet to be executed.

% Some examples of rules in single-step semantics 
% would be as follows:
% \[ \infer {
%     \opsem{a}{\sigma} \downarrow n}
%   {\opsem{x \coloneqq    a}{\sigma} 
%     \to \opsem{skip}{\sigma[x \mapsto a]}} \]
% \[ \infer {
%     \opsem{c$_1$}{\sigma} \to \opsem{c$_1$'}{\sigma'} \\
%     \imp{c$_1$'} \neq \imp{skip}}
%    {\opsem{c$_1$; c$_2$}{\sigma}
%      \to \opsem{c$_1$'; c$_2$}{\sigma'}} \]
% \[ \infer {
%     \opsem{c$_1$}{\sigma} \to \opsem{skip}{\sigma'}}
%    {\opsem{c$_1$; c$_2$}{\sigma}
%      \to \opsem{c$_2$}{\sigma'}} \]

% \begin{example}
%     For the program \imp{c = x \coloneqq    0; x \coloneqq    1}, we have:
%     \begin{align*}
%       \opsem{x \coloneqq    0}{\sigma} 
%       & \to \opsem{skip}{\sigma[x \mapsto 0]} \\
%       \opsem{x \coloneqq    0; x \coloneqq    1}{\sigma} 
%       & \to \opsem{x \coloneqq    1}{\sigma[x \mapsto 0]} \\
%       & \to \opsem{skip}{\sigma[x \mapsto 1]}
%     \end{align*}
% \end{example}



\subsection{Denotational Semantics}

\begin{definition}[Denotational Semantics]
We define functions 
$\sde{a}: \Sigma \to \ZN$ and 
$\sde{b}: \Sigma \to \BN$ 
as semantic functions for arithmetic and boolean expressions.
For command expressions, we define 
$\sde{c}: \Sigma \to \Sigma_\bot$ 
% where $\bot$ denotes non-termination.
(alternatively define $\sde{c}$ as the \emph{partial} function $\Sigma \rightharpoonup \Sigma$).
Similar to operational semantics, 
we define denotational semantic functions 
for different types of expressions in IMP:
\begin{enumerate}
    \item Arithmetic expressions:
    \begin{align*}
        \sde{n} &\coloneqq \defun{n}\\
        \sde{x} &\coloneqq \defun{\sigma(x)}  \\
        \sde{a_1 + a_2} &\coloneqq
          \defun{\sde{a_1} + \sde{a_2}}
    \end{align*}
    Notice the difference between the syntactic $+$ in $  \sde{a_1 + a_2}$ and the semantic one  $\defun{\sde{a_1} + \sde{a_2}}$
    the syntactic + and the semantic `+' operators.
    \item Boolean expressions:
    \begin{align*}
        \sde{true}  &\coloneqq \defun{\top} \\
        \sde{a_1 \leq a_2}  &\coloneqq 
          \defun{\sde{a_1} \sigma \leq \sde{a_2} \sigma}  \\ 
        \sde{b_1 \land b_2}  &\coloneqq 
          \defun{\sde{b_1} \sigma \land \sde{b_2} \sigma}
    \end{align*}
    Again, notice the difference between the syntactic and semantic $\land$.
    \item Command expressions:
    \begin{align*}
        \sde{\sskip}  &\coloneqq \defun{\sigma} \\
        \sde{x \coloneqq    a}  &\coloneqq
          \defun{\sigma[x \mapsto \sde{a} \sigma]} \\
        \sde{c_1; c_2}  &\coloneqq 
          \defun{\sif{\sde{c_1} \sigma = \bot}{\bot}{\sde{c_2}(\sde{c_1} \sigma)}} \\
        \sde{ \sif{b}{c_1}{c_2}} &\coloneqq 
          \defun{ \sif{\sde{b} \sigma = \top}{\sde{c_1} \sigma}{\sde{c_2} \sigma}}
    \end{align*}
    The case for a $\swhile{a}{b}$ command is more elaborate:
    its semantic function is defined as 
    the \emph{least fixed point} of a higher order function $F$,
    which takes a program $f: \Sigma \to \Sigma_\bot$
    and a state $\sigma$, and returns a state $\sigma' \in \Sigma_\bot$.
    We formally define function $F$ as follows: 
    \begin{align*}
         F(f) = 
      \defun{\sif{\sde{b} \sigma = \top}{(\sif{\sde{c} \sigma = \bot}{\bot}{f(\sde{c} \sigma)})}{\sigma}}
    \end{align*}
\end{enumerate}
\end{definition}


\begin{exercise}
    We can approximate the lfp of a function $F$ 
    by constructing the sequence 
    $F(\bot), F^2(\bot), \dots$.
    For the factorial program,
    first \emph{define} a program $\hat{\bot}$,
    then calculate 
    $\hat{F}(\hat{\bot}), \hat{F}^2(\hat{\bot}), \dots$,
    where $\hat{F}$ is 
    the semantic function of the $\swhile{a}{b}$ command in factorial.
\end{exercise}

The following theorem connects 
operational and denotational semantics:

\begin{theorem}[Connect Operational and Denotational Semantic]
  The following equivalences hold
  over expressions in IMP:
  \begin{alignat*}{5}
    & \forall \simp{a}, n, \sigma && .~ 
      \sop{a}{\sigma} \downarrow n 
      &&& \iff \sde{a} \sigma & = n \\
    & \forall \simp{b}, n, \sigma && .~
      \sop{a}{\sigma} \downarrow v
      &&& \iff \sde{b} \sigma & = v \\
    & \forall \simp{c}, \sigma, \sigma' && .~
      \sop{c}{\sigma} \downarrow \sigma'
      &&& \iff \sde{c} \sigma & = \sigma' \qquad (*) \\
    & \forall \simp{c}, \sigma && .~
      (\neg \exists \sigma'.~ \sop{c}{\sigma} \downarrow \sigma')
      &&& \iff \sde{c} \sigma & = \bot
  \end{alignat*}
  Arithmetic, Boolean, and command expressions
  are denoted with $\simp{a}$, $\simp{b}$, and $\simp{c}$.
\end{theorem}

\begin{exercise}[Optional]
    Prove (*) in the theorem above.
    You can assume the equivalences for 
    arithmetic and Boolean expressions.
\end{exercise}


\subsection{Axiomatic Semantics (Hoare Logic)}


\begin{definition}[Notation]
Let $\simp{c}$ be an IMP program, let $\sigma \in \Sigma$ is a state,
and $\interp: Y \mapsto \ZN$ is a mapping of free variables to integers,
then
\begin{align*}
    \sigma \models^{\interp}  \{ \varphi \} \;\simp{c}\;  \{ \psi \} 
\end{align*}
denotes that the program $\simp{c}$ satisfies the assertions in $\{ \psi \}$, if the assertions in $\{ \varphi \}$ are satisfied before execution.
Each of $\{ \varphi \}$ and $\{ \psi \}$
are a set of \emph{assertions} 
($\sim$ first-order logic formulas)
over the set of program locations $X$,
the set of free variables $Y$,
and at least all functions and predicate symbols
that occur in arithmetic and Boolean expressions of IMP.
\end{definition}

% We use the following notation to show 
% program \imp{c} satisfies assertions in $\{ \psi \}$,
% if assertions in $\{ \varphi \}$ are satisfied before execution:
% \[ \sigma \models^I  \{ \varphi \} ~\imp{c}~ \{ \psi \} \]
% where $\sigma \in \Sigma$ is a state, 
% \imp{c} is an IMP program,
% and $I: Y \mapsto \ZN$ is 
% a mapping of free variables to integers.



\begin{definition}[Partial Correctness]
    The expression $\models^{\interp}$ is defined as follows.
    $\sigma \models^{\interp} \{ \varphi \} ~\simp{c}~ \{ \psi \}$
    if and only if for any $\sigma'$, 
    if $\sop{c}{\sigma} \downarrow \sigma'$
    and $\sigma \models^{\interp} \varphi$, 
    then $\sigma' \models^{\interp} \psi$.
    Moreover, we write $\models \{ \varphi \} ~\simp{c}~ \{ \psi \}$ iff 
    $\forall \sigma, I.~ 
      \sigma \models^I \{ \varphi \} ~\simp{c}~ \{ \psi \}$
\end{definition}

\begin{definition}[Axiomatic Semantics]
 Similar to other proof systems, 
axiomatic semantics for IMP is a set of inference rules.
% We begin with defining the axioms for $\simp{\sskip}$ and assignment:
\begin{enumerate}
    \item Skip and assignment command
    \begin{center}
    \AxiomC{}
    \UnaryInfC{$\{\varphi \} ~\simp{\sskip}~ \{ \varphi \}$}
    \DisplayProof
    $\quad$
    \AxiomC{$ \varphi[x \mapsto a]$}
    \UnaryInfC{$\simp{x \coloneqq    a}~ \{ \varphi \}$}
    \DisplayProof
    $\quad$
\end{center}
\item Command expressions
    \begin{center}
    \AxiomC{$\{\varphi \} ~\simp{c_1}~ \{ \psi \}$}
    \AxiomC{$\{ \psi \} ~\simp{c_2}~ \{ \chi \} $}
    \BinaryInfC{$\{ \varphi \} ~\simp{c_1; c_2}~ \{ \chi \} $}
    \DisplayProof
    $\quad$
    \AxiomC{$ \{ \varphi \land b \} ~\simp{c_1}~ \{ \psi \} $}
    \AxiomC{$\{ \varphi \land \neg b \} ~\simp{c_2}~ \{ \psi \} $}
    \BinaryInfC{$\{ \varphi \} 
    ~\simp{\sif{b}{c_1}{c_2}}~  \{ \psi \}$}
    \DisplayProof
\end{center}

\item Loop invariant
    \begin{center}
    \AxiomC{$\{ \varphi \land b \} ~\simp{c}~ \{ \varphi \}$}
    \UnaryInfC{$\{ \varphi \} 
    ~\simp{\swhile{b}{c}}~ 
    \{ \varphi \land \neg b \}$}
    \DisplayProof
\end{center}
\item The consequence rule, it shows how we can
\emph{strengthen} our pre-conditions
and \emph{weaken} our post-conditions:
  \begin{center}
    \AxiomC{$ \varphi \simplies \varphi' $}
    \AxiomC{$\{ \varphi' \} ~\simp{c}~ \{ \psi' \}$}
    \AxiomC{$  \psi' \simplies \psi $}
    \TrinaryInfC{$\{ \varphi \} ~\simp{c}~ \{ \psi \} $}
    \DisplayProof
\end{center}
\end{enumerate}

\end{definition}

\begin{remark}
    Be careful with the pre- and post-conditions of the assignment command.
\end{remark}


\begin{exercise}
    For the factorial program $\simp{fact}$,
    prove the following using Hoare logic:$ \{ x = i \} ~\simp{fact}~ \{ y = i! \}$
\end{exercise}





% We define functions 
% $\sde{a}: \Sigma \to \ZN$ and 
% $\sde{b}: \Sigma \to \BN$ 
% as semantic functions for arithmetic and boolean expressions.
% For command expressions, we define 
% $\desem{c}: \Sigma \to \Sigma_\bot$, 
% where $\bot$ denotes non-termination.
% We could also define $\desem{c}$
% as a \emph{partial} function $\Sigma \rightharpoonup \Sigma$.

% Similar to operational semantics, 
% we define denotational semantic functions 
% for different types of expressions in IMP:
% \begin{enumerate}
%     \item Arithmetic expressions:
%     \begin{align*}
%         \desem{n} & = \defun{n} \\
%         \desem{x} & = \defun{\sigma(x)} \\
%         \desem{a$_1$ + a$_2$} &= 
%           \defun{\desem{a$_1$} + \desem{a$_2$}}
%     \end{align*}
%     Notice the difference between 
%     the syntactic `\imp{+}' and the semantic `+' operators.
%     \item Boolean expressions:
%     \begin{align*}
%         \desem{true} & = \defun{\top} \\
%         \desem{a$_1$ $\leq$ a$_2$} & = 
%           \defun{\desem{a$_1$} \sigma \leq \desem{a$_2$} \sigma} \\
%         \desem{b$_1$ $\land$ b$_2$} & = 
%           \defun{\desem{b$_1$} \sigma \land \desem{b$_2$} \sigma}
%     \end{align*}
%     Again, notice the difference between the two $\land$'s.
%     \item Command expressions:
%     \begin{align*}
%         \desem{skip} & = \defun{\sigma} \\
%         \desem{x \coloneqq    a} & = 
%           \defun{\sigma[x \mapsto \desem{a} \sigma]} \\
%         \desem{c$_1$; c$_2$} & = 
%           \defun{\text{
%             if $\desem{c$_1$} \sigma = \bot$ then $\bot$
%             else $\desem{c$_2$}(\desem{c$_1$} \sigma)$ }} \\
%         \desem{if b then c$_1$ else c$_2$} & = 
%           \defun{\text{
%             if $\desem{b} \sigma = \top$ then $\desem{c$_1$} \sigma$
%             else $\desem{c$_2$} \sigma$ }}
%     \end{align*}
%     The case for a \imp{while} command is more elaborate:
%     its semantic function is defined as 
%     the \emph{least fixed point} of a higher order function $F$,
%     which takes a program $f: \Sigma \to \Sigma_\bot$
%     and a state $\sigma$, and returns a state $\sigma' \in \Sigma_\bot$.
%     We formally define function $F$ as follows: 
%     \[ F(f) = 
%       \defun{\text{
%         if $\desem{b} \sigma = \top$ then 
%          (if $\desem{c} \sigma = \bot$ then $\bot$
%           else $f(\desem{c} \sigma)$) 
%         else $\sigma$
%       }} \]
% \end{enumerate}

% \begin{exercise}
%     Remember that we can 
%     approximate the lfp of a function $F$ 
%     by constructing the sequence 
%     $F(\bot), F^2(\bot), \dots$.
%     For the factorial program,
%     first \emph{define} a program $\hat{\bot}$,
%     then calculate 
%     $\hat{F}(\hat{\bot}), \hat{F}^2(\hat{\bot}), \dots$,
%     where $\hat{F}$ is 
%     the semantic function of the \imp{while} command in factorial.
% \end{exercise}

% The following theorem connects 
% operational and denotational semantics:
% \begin{theorem} 
%   The following equivalences hold
%   over expressions in IMP:
%   \begin{alignat*}{5}
%     & \forall \imp{a}, n, \sigma && .~ 
%       \opsem{a}{\sigma} \downarrow n 
%       &&& \iff \desem{a} \sigma & = n \\
%     & \forall \imp{b}, n, \sigma && .~
%       \opsem{a}{\sigma} \downarrow v
%       &&& \iff \desem{b} \sigma & = v \\
%     & \forall \imp{c}, \sigma, \sigma' && .~
%       \opsem{c}{\sigma} \downarrow \sigma'
%       &&& \iff \desem{c} \sigma & = \sigma' \qquad (*) \\
%     & \forall \imp{c}, \sigma && .~
%       (\neg \exists \sigma'.~ \opsem{c}{\sigma} \downarrow \sigma')
%       &&& \iff \desem{c} \sigma & = \bot
%   \end{alignat*}
%   Arithmetic, Boolean, and command expressions
%   are denoted with \imp{a}, \imp{b}, and \imp{c}.
% \end{theorem}

% \begin{exercise}[Optional]
%     Prove (*) in the theorem above.
%     You can assume the equivalences for 
%     arithmetic and Boolean expressions.
% \end{exercise}

% \section{Axiomatic Semantics (Hoare Logic)}

% We use the following notation to show 
% program \imp{c} satisfies assertions in $\{ \psi \}$,
% if assertions in $\{ \varphi \}$ are satisfied before execution:
% \[ \sigma \models^I  \{ \varphi \} ~\imp{c}~ \{ \psi \} \]
% where $\sigma \in \Sigma$ is a state, 
% \imp{c} is an IMP program,
% and $I: Y \mapsto \ZN$ is 
% a mapping of free variables to integers.
% Each of $\{ \varphi \}$ and $\{ \psi \}$
% are a set of \emph{assertions} 
% ($\sim$ first-order logic formulas)
% over the set of program locations $X$,
% the set of free variables $Y$,
% and at least all functions and predicate symbols
% that occur in arithmetic and Boolean expressions of IMP.
% The following definition formalizes 
% how $\models^I$ should be interpreted:

% \begin{definition}[Partial Correctness]
%     $\sigma \models^I \{ \varphi \} ~\imp{c}~ \{ \psi \}$
%     if and only if for any $\sigma'$, 
%     if $\opsem{c}{\sigma} \downarrow \sigma'$
%     and $\sigma \models^I \varphi$, 
%     then $\sigma' \models^I \psi$.
% \end{definition}

% \begin{definition}
%     $\models \{ \varphi \} ~\imp{c}~ \{ \psi \}$ iff 
%     $\forall \sigma, I.~ 
%       \sigma \models^I \{ \varphi \} ~\imp{c}~ \{ \psi \}$
% \end{definition}

% Similar to other proof systems, 
% axiomatic semantics for IMP is a set of inference rules.
% We begin with defining the axioms for \imp{skip} and assignment:
% \[ \infer 
%   { }{ \{ \varphi \} ~\imp{skip}~ \{ \varphi \} } 
% \qquad \infer 
%   { }{ \{ \varphi[x \mapsto a] \} 
%     ~\imp{x \coloneqq    a}~ \{ \varphi \} } \]
% Be careful with the pre- and post-conditions 
% of the assignment command.

% For the rest of the command expressions,
% we define the following rules:
% \[ \infer
%   { \{ \varphi \} ~\imp{c$_1$}~ \{ \psi \} \\ 
%     \{ \psi \} ~\imp{c$_2$}~ \{ \chi \} }
%   { \{ \varphi \} ~\imp{c$_1$; c$_2$}~ \{ \chi \} } \]
% \[ \infer
%   { \{ \varphi \land b \} ~\imp{c$_1$}~ \{ \psi \} \\ 
%     \{ \varphi \land \neg b \} ~\imp{c$_2$}~ \{ \psi \} }
%   { \{ \varphi \} 
%     ~\imp{if b then c$_1$ else c$_2$}~ 
%     \{ \psi \} }
% \]

% In the next rule, 
% $\varphi$ is often referred to 
% as the \emph{loop invariant}:
% \[ \infer
%   { \{ \varphi \land b \} ~\imp{c}~ \{ \varphi \} }
%   { \{ \varphi \} 
%     ~\imp{while b do c}~ 
%     \{ \varphi \land \neg b \} } \]

% The following rule,
% also known as \textsc{Consequence},
% shows how we can
% \emph{strengthen} our pre-conditions
% and \emph{weaken} our post-conditions:
% \[ \infer
%   { \varphi \implies \varphi' \\
%     \{ \varphi' \} ~\imp{c}~ \{ \psi' \} \\
%     \psi' \implies \psi }
%   { \{ \varphi \} ~\imp{c}~ \{ \psi \} } \]

% \begin{exercise}
%     For the factorial program \imp{fact},
%     prove the following using Hoare logic:
%     \[ \{ x = i \} ~\imp{fact}~ \{ y = i! \} \]
% \end{exercise}

