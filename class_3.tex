{{%Localize command definitions

\newcommand{\NN}{\mathbb{N}}
\renewcommand{\implies}{\Rightarrow}
\newcommand{\implied}{\Leftarrow}
\newcommand{\eps}{\varepsilon}
\newcommand{\gfp}{\mathrm{gfp}}
\newcommand{\lfp}{\mathrm{lfp}}
\newcommand{\defas}{\coloneqq}
\renewcommand{\le}{\sqsubseteq}
\renewcommand{\ge}{\sqsupseteq}
\renewcommand{\sqcap}{\bigsqcap}
\renewcommand{\sqcup}{\bigsqcup}
\newcommand{\bottom}{\perp}


\newcommand{\Theorem}[2][]{\begin{theorem}[#1]#2\end{theorem}}
\newcommand{\Definition}[2][]{\begin{definition}[#1]#2\end{definition}}
\newcommand{\Example}[2][]{\begin{example}[#1]#2\end{example}}
\newcommand{\Homework}[2][]{\begin{homework}[#1]#2\end{homework}}
\newcommand{\Lemma}[2][]{\begin{lemma}[#1]#2\end{lemma}}


\chapter{Class 3}

	

\Definition[Prefixpoint]{
	Consider a lattice $(A, \le)$ and a function $f \colon A \to A$. 
	The set of prefixes is
	\[
		\{ x \in A : x \le f(x) \} \,.
	\] 
}

\Definition[Postfixpoint]{
	Consider a lattice $(A, \le)$ and a function $f \colon A \to A$. 
	The set of postfixes is
	\[
		\{ x \in A : f(x) \le x \} \,.
	\] 
}

\Definition[$\gfp$ and $\lfp$]{
	Consider a complete lattice $(A, \le)$ and a function $f \colon A \to A$. 
	Then, 
	\begin{align*}
		\gfp f &\defas \sqcup \{ x \in A : x \le f(x) \} \\
		\lfp f &\defas \sqcap \{ x \in A : f(x) \le x \} \,.
	\end{align*}
}

\Theorem[Fixpoints] {
	Consider a complete lattice $(A, \le)$ and a monotonic function $f \colon A \to A$.
	Then, $\gfp f$ and $\lfp f$ are fixpoints of $f$ and, for all fixpoints $x$ of $f$, we have $\lfp f \le x \le \gfp f$.
}


\Definition[$\sqcup$-continuous]{
	Consider a complete lattice $(A, \le)$.
	A function $f \colon A \to A$ is $\sqcup$-continuous if, for all increasing sequences $x_0 \le x_1 \le x_2 \le x_2 \le \ldots$, we have
	\[
		f \left( \sqcup \{ x_n : n \in \NN \} \right) = \sqcup \{ f(x_n) : n \in \NN \} \,.
	\] 
}

\Definition[$\sqcap$-continuous]{
	Consider a complete lattice $(A, \le)$.
	A function $f \colon A \to A$ is $\sqcap$-continuous if, for all increasing sequences $x_0 \ge x_1 \ge x_2 \ge x_2 \ge \ldots$, we have
	\[
		f \left( \sqcap \{ x_n : n \in \NN \} \right) = \sqcap \{ f(x_n) : n \in \NN \} \,.
	\] 
}

\Lemma{
	$\sqcup$-continuous implies monotonicity and $\sqcap$-continuous implies monotonicity.
}


\Theorem[Constructive fixpoints] {
	Consider a complete lattice $(A, \le)$ and a monotonic function $f \colon A \to A$.
	Then, 
	\begin{align*}
		\lfp f 
			&= \sqcup \{ f^n(\bottom) : n \in \NN \} \\
		\gfp f 
			&= \sqcap \{ f^n(\top) : n \in \NN \} \,.
	\end{align*}
}

\Homework{
	Prove this theorem.
}


\Definition[$\NN$]{
	Define $\NN$ as the smallest set X such that 
	\begin{enumerate}
		\item $0 \in X$
		\item if $n \in X$, then $S n \in X$
	\end{enumerate} 
}

In the definition of $\NN$, we consider a universal set $U$ sufficiently big, the complete lattice $(2^U, \subseteq)$ and the function on sets given by $f(Y) \defas \{0\} \cup \{ S n : n \in Y \}$. 
Then, $\lfp f = \NN$.	

\Definition[Set of words]{
	Consider a finite alphabet $\Sigma$.
	Define $\Sigma^*$ as the smallest set $X$ such that 
	\begin{enumerate}
		\item $\eps \in X$
		\item for all $a \in \Sigma$, we have $aX \subseteq X$.
	\end{enumerate} 
}

A few remarks are in place.
\begin{itemize}
	\item 
		Inductively defined sets are countable and consists of finite elements.
	\item 
		Inductively defined sets can be written as rules $x \implies f(x)$ meaning that, if $x \in X$, then $f(x) \in X$.
	\item 
		Inductively defined sets allow proof by induction.
		Consider prove that for all $x \in X$ we have $G(x)$.
		This can be proven by showing
		\begin{enumerate}
			\item $G(\bottom)$
			\item For all $x \in X$, if $G(x)$, then $G(f(x))$
		\end{enumerate}
\end{itemize}

\Definition[Balanced binary sequences]{
	Define the set $S$ as the largest set $X$ such that
	\begin{enumerate}
		\item $X \subseteq 01X \cup 10X$.
	\end{enumerate}
}

In the definition of balanced binary sequences, we consider the complete lattice $(\Sigma^\omega, \subseteq)$ and the function on sets given by $f(X) \defas 01X \cup 10X$. 
Then, balanced binary sequences corresponds to $\gfp f$.	

\Definition[Interval {$[0, 1]$}]{
	Define the set $S$ as the largest set $X$ such that
	\begin{enumerate}
		\item $X \subseteq 0X \cup 1X \cup \ldots \cup 9X$.
	\end{enumerate}
}

A few remarks are in place.
\begin{itemize}
	\item 
		Coinductively defined sets are uncountable and consists of infinite elements.
	\item 
		Coinductively defined sets can be written as rules $x \implied f(x)$ meaning that, for all $y \in X$, there exists $x$ such that $y = f(x)$ and $x \in X$.
	\item 
		Coinductively defined sets allow proof by induction.
		Consider prove that for all $x \in X$ we have $G(x)$.
		This can be proven by showing
		\begin{enumerate}
			\item For all $x$ and $i$, if $G(f_i(x))$, then $G(x)$
		\end{enumerate}
		where $\{f_1, \ldots, f_n\}$ is the set of rules that define the set $X$.
\end{itemize}


\Homework[Prove balanced binary sequences]{
	Consider $S$ generated by the rules $X \implied 01X$ and $X \implied 10X$. 
	Prove that, for all binary words $x$, we have that $x\ in S$ if and only if every finite prefix of even length of $x$ has the same number of $0$s and $1$s.  
}

Hints.
\begin{enumerate}
	\item The direction $\implied$ can be proven by coinduction.
	\item The direction $\implies$ can be proven by induction on the length of the prefix.
\end{enumerate}

}} % End localization of command definitions
