

% \begin{definition}[Prefixpoint]
%     Consider a lattice $(A, \sqsubseteq)$ and a function $f \colon A \to A$. 
% 	The set of prefixes is
% 	\begin{align*}
% 	    	\{ x \in A : x \sqsubseteq f(x) \} \,.
% 	\end{align*}
% \end{definition}


% \begin{definition}[Postfixpoint]
%     Consider a lattice $(A, \sqsubseteq)$ and a function $f \colon A \to A$. 
% 	The set of postfixes is
%  \begin{align*}
% \{ x \in A : f(x) \sqsubseteq x \} \,.
% 	\end{align*}
% \end{definition}

% \begin{definition}[$\gfp$ and $\lfp$]
%     Consider a complete lattice $(A, \sqsubseteq)$ and a function $f \colon A \to A$. 
% 	Then, 
% 	\begin{align*}
% 		\gfp f &\coloneqq \bigsqcup \{ x \in A : x \sqsubseteq f(x) \} \\
% 		\lfp f &\coloneqq \bigsqcap \{ x \in A : f(x) \sqsubseteq x \} \,.
% 	\end{align*}
% \end{definition}

%% \begin{definition}[Pre- and Post-fixpoint]
%%     Consider a lattice $(A, \sqsubseteq)$ and a function $f \colon A \to A$.
%% 	The set of pre-fixpoint is $\{ x \in A : x \sqsubseteq f(x) \} \,$ and the set of post-fixpoint is $\{ x \in A : f(x) \sqsubseteq x \} \,$  where $x$ is a pre-fixpoint and post-fixpoint respectively.
%% \end{definition}


%% \begin{definition}[$\gfp$ and $\lfp$]
%%     Consider a complete lattice $(A, \sqsubseteq)$ and a function $f \colon A \to A$.
%% 	Then, $\gfp f \coloneqq \bigsqcup \{ x \in A : x \sqsubseteq f(x) \}$ and $\lfp f \coloneqq \bigsqcap \{ x \in A : f(x) \sqsubseteq x \}$
%% \end{definition}



%% \begin{theorem}[Knaster Tarski (Version 2)]
%%     Consider a complete lattice $(A, \sqsubseteq)$ and a monotonic function $f \colon A \to A$.
%% 	Then, $\gfp f$ and $\lfp f$ are fixpoints of $f$ and, for all fixpoints $x$ of $f$, we have $\lfp f \sqsubseteq x \sqsubseteq \gfp f$.
%% \end{theorem}

\begin{definition}[$\bigsqcup$-continuous]
    Consider a complete lattice $(A, \sqsubseteq)$.
	A function $f \colon A \to A$ is $\bigsqcup$-continuous if, for all increasing sequences $x_0 \sqsubseteq x_1 \sqsubseteq x_2 \sqsubseteq x_2 \sqsubseteq \ldots$, we have
    \begin{align*}
        f \left( \bigsqcup \{ x_n : n \in \NN \} \right) = \bigsqcup \{ f(x_n) : n \in \NN \} \,.
    \end{align*}
\end{definition}


\begin{definition}[$\bigsqcap$-continuous]
    Consider a complete lattice $(A, \sqsubseteq)$.
	A function $f \colon A \to A$ is $\bigsqcap$-continuous if, for all increasing sequences $x_0 \sqsupseteq x_1 \sqsupseteq x_2 \sqsupseteq x_2 \sqsupseteq \ldots$, we have
 \begin{align*}
     	f \left( \bigsqcap \{ x_n : n \in \NN \} \right) = \bigsqcap \{ f(x_n) : n \in \NN \} \,.
 \end{align*}
\end{definition}

\begin{lemma}
    Both $\bigsqcup$-continuous and $\bigsqcap$-continuous imply monotonicity respectively. 
\end{lemma}



\begin{theorem}[Constructive Fixpoints]
\label{thrm:constructive-fixedpoints}
    Consider a complete lattice $(A, \sqsubseteq)$ and a monotonic function $f \colon A \to A$.
	Then, 
	\begin{align*}
		\lfp f 
			= \bigsqcup \{ f^n(\bottom) : n \in \NN \}  \quad \text{and} \quad 
		\gfp f 
			= \bigsqcap \{ f^n(\top) : n \in \NN \} \,.
	\end{align*}
\end{theorem}


\begin{exercise}
    Prove the Constructive Fixedpoints Theorem (Theorem \ref{thrm:constructive-fixedpoints})
\end{exercise}
	

\subsection{Induction and Co-Induction}

\begin{definition}
\label{def:natural-numbers}
    Define $\NN$ as the smallest set X s.t.\ (i) $0 \in X$ and (ii) if $n \in X$, then $S n \in X$
\end{definition}

\begin{remark}
    In Definition \ref{def:natural-numbers}, we consider a universal set $U$ sufficiently big, the complete lattice $(2^U, \subseteq)$ and the function on sets given by $f(Y) \coloneqq \{0\} \cup \{ S n : n \in Y \}$. 
Then, $\lfp f = \NN$.	
\end{remark}



\begin{definition}
    Consider a finite alphabet $\Sigma$.
	Define $\Sigma^*$ as the smallest set $X$ such that (i) $\emptystring \in X$ and (ii) for all $a \in \Sigma$, we have $aX \subseteq X$.
\end{definition}

\begin{remark}
    Inductively defined sets are countable and consist of finite elements.
    They can be written as rules of the form
\begin{prooftree}
   \AxiomC{$x$}
   \UnaryInfC{$f(x)$}
\end{prooftree}
    expressing that if $x \in X$, then $f(x) \in X$. Moreover, they allow proof by induction. 
    \begin{quote}
         Consider proving that for all $x \in X$ we have $G(x)$. This can be proven by showing (i) $G(\bottom)$ and (ii) for all $x \in X$, if $G(x)$, then $G(f(x))$.
    \end{quote}
\end{remark}

\begin{definition}[Balanced binary sequences]
    Define the set $S$ as the largest set $X$ such that $X \subseteq 01X \cup 10X$.
\end{definition}

\begin{remark}
    In the definition of balanced binary sequences, we consider the complete lattice $(\Sigma^\omega, \subseteq)$ and the function on sets given by $f(X) \coloneqq 01X \cup 10X$. Then, balanced binary sequences correspond to $\gfp f$.	
\end{remark}


\begin{definition}[Interval {$[0, 1]$}]
    Define the set $S$ as the largest set $X$ such that $X \subseteq 0X \cup 1X \cup \ldots \cup 9X$.
\end{definition}


\begin{remark}
    Co-inductively defined sets are uncountable and consist of infinite elements.
    They can be written as rules of the form
    \begin{prooftree}
   \AxiomC{$x$}
   \UnaryInfC{$f(x)$}
\end{prooftree}
   expressing that for all $y \in X$, there exists $x$ such that $y = f(x)$ and $x \in X$. Moreover, they allow proof by co-induction.
    \begin{quote}
        Consider proving that for all $x$, if $G(x)$, then $x \in X$.
		This can be proven by showing (i) for all $x$ and $i$, if $G(f_i(x))$, then $G(x)$ \,, where $\{f_1, \ldots, f_n\}$ is the set of rules that define the set $X$.
    \end{quote}
\end{remark}



\begin{exercise}[Prove balanced binary sequences]
    Consider $S$ generated by the rules $X \simplied 01X$ and $X \simplied 10X$. 
	Prove that, for all binary words $x$, we have that $x\ in S$ if and only if every finite prefix of even length of $x$ has the same number of $0$s and $1$s.  
(Hints: The direction $\simplied$ can be proven by co-induction. The direction $\simplies$ can be proven by induction on the length of the prefix.)
\end{exercise}


