{{%Localize command definitions

\chapter{Class 8}

We now talk about three proof systems:
\begin{itemize}
\item Hilbert $$\vdash \phi$$
\item Natural deduction $$\Gamma \vdash \phi$$
\item Grentzen $$\Gamma \vdash \Delta$$
\end{itemize}
We now talk abou three decision procedures (either for validity or for satisfiability):
\begin{itemize}
\item Branching (if we don't derive $\bot$ then it is satisfiable)
$$\inferrule
{\Gamma[\bot]\qquad\qquad\Gamma[\top]}
{\Gamma[p]}
$$
\item Resolution
$$\inferrule
{\Gamma,\, C_1,\, C_2,\, C_1[\bot] \vee C_2[\bot]}
{\Gamma,\, C_1[p],\, C_2[p]}
$$
\item Unit resolution, combined with branching of lower priority (DPLL)
$$\inferrule
{\Gamma,\, C[\bot]}
{\Gamma,\, \ell,\, C[\neg\ell]}
$$
Example: $$p \vee q \vee r ,\; \neg p \vee \neg q \vee ¬r ,\; \neg p \vee q \vee r ,\; \neg q \vee r ,\; q \vee \neg r$$
First we branch on $r$.
TODO.
It is satisfied by $p=\bot,\; q=\top,\; r=\top$.
\end{itemize}

Horn clause is a clause with at most one positive literal.
We can view them as implications where LHS is a conjunction of positive propositions:
\begin{align*}
\neg p \vee \neg q 
	&\iff p \wedge q \rightarrow \bot \\
\neg p \vee \neg q \vee r 
	&\iff p \wedge q \rightarrow r \\
r 
	&\iff \top \rightarrow r \\
\end{align*}

\subsection{Metatheorems}

\subsubsection{Compactness}

Countable set $\Gamma$ of formulas is satisfiable iff every finite subset of $\Gamma$ is satisfiable.

\subsubsection{Craig's interpolation}

\subsubsection{Cut elimination}

Cut rule in NK / NJ:
$$\inferrule
{\Gamma\vdash\phi\qquad\qquad\qquad\Gamma,\phi\vdash\psi}
{\Gamma\vdash\psi}
$$
Cut rule in LK:
$$\inferrule
{\Gamma\vdash\phi,\Delta\qquad\qquad\qquad\Gamma,\phi\vdash\Delta}
{\Gamma\vdash\Delta}
$$

If a judgement can be proved with the cut rule, it can also be proved without the cut rule.
In fact, it is ``iff''; the other direction is trivial.
Of course, the proof may become longer (introducing a lemma $\phi$ often helps in practice).
It is a purely syntactic metatheorem. We cannot use deduction to prove it.

\section{First-order logic}

First-order logic, also called ``predicate logic'', is propositional logic with quantifiers $\forall$ (for all) and $\exists$ (exists).

\subsection{Syntax}

\subsubsection{Nonlogical symbols (``signature'')}

\subsubsection{Logical symbols}

\subsubsection{Safe substitution}

}} % End localization of command definitions
