{{%Localize command definitions

\chapter{Class 8}

We now talk about three proof systems:
\begin{itemize}
\item Hilbert $$\vdash \phi$$
\item Natural deduction $$\Gamma \vdash \phi$$
\item Grentzen $$\Gamma \vdash \Delta$$
\end{itemize}
We now talk abou three decision procedures (either for validity or for satisfiability):
\begin{itemize}
\item Branching (if we don't derive $\bot$ then it is satisfiable)
$$\inferrule
{\Gamma[\bot]\qquad\qquad\Gamma[\top]}
{\Gamma[p]}
$$
\item Resolution
$$\inferrule
{\Gamma,\, C_1,\, C_2,\, C_1[\bot] \vee C_2[\bot]}
{\Gamma,\, C_1[p],\, C_2[p]}
$$
\item Unit resolution, combined with branching of lower priority (DPLL)
$$\inferrule
{\Gamma,\, C[\bot]}
{\Gamma,\, \ell,\, C[\neg\ell]}
$$
Example: $$p \vee q \vee r ,\; \neg p \vee \neg q \vee \neg r ,\; \neg p \vee q \vee r ,\; \neg q \vee r ,\; q \vee \neg r$$
First we branch on $r$.
\begin{itemize}
\item Case $r=\bot$:
$$p \vee q ,\; \neg p \vee q ,\; \neg q$$
We didn't continue this branch.
\item Case $r=\top$:
$$\neg p \vee \neg q ,\; q$$
Then we propagate $q$. We end up with $\neg p$.\\
It is satisfied by $p=\bot,\; q=\top,\; r=\top$.
\end{itemize}
\end{itemize}

Horn clause is a clause with at most one positive literal.
We can view them as implications where LHS is a conjunction of positive propositions:
\begin{align*}
\neg p \vee \neg q 
	&\iff p \wedge q \rightarrow \bot \\
\neg p \vee \neg q \vee r 
	&\iff p \wedge q \rightarrow r \\
r 
	&\iff \top \rightarrow r \\
\end{align*}

\subsection{Metatheorems}

\subsubsection{Compactness}

Countable set $\Gamma$ of formulas is satisfiable iff every finite subset of $\Gamma$ is satisfiable.

\subsubsection{Craig's interpolation}

We have $\vdash \phi \rightarrow \psi$ iff there exists a third formula $\chi$ (the "interpolant") which only uses
nonlogical symbols (in propositional logic, it is propositions only) that occur in
both $\phi$ and $\psi$ such that $\vdash \phi \rightarrow \chi$ and $\vdash \chi \rightarrow \psi$.

\subsubsection{Cut elimination}

Cut rule in NK / NJ:
$$\inferrule
{\Gamma\vdash\phi\qquad\qquad\qquad\Gamma,\phi\vdash\psi}
{\Gamma\vdash\psi}
$$
Cut rule in LK:
$$\inferrule
{\Gamma\vdash\phi,\Delta\qquad\qquad\qquad\Gamma,\phi\vdash\Delta}
{\Gamma\vdash\Delta}
$$

If a judgement can be proved with the cut rule, it can also be proved without the cut rule.
In fact, it is ``iff''; the other direction is trivial.
Of course, the proof may become longer (introducing a lemma $\phi$ often helps in practice).
It is a purely syntactic metatheorem. We cannot use deduction to prove it.

\section{First-order logic}

First-order logic, also called ``predicate logic'', is propositional logic with quantifiers $\forall$ (for all) and $\exists$ (exists).

\subsection{Syntax}

\subsubsection{Nonlogical symbols (``signature'')}
\begin{itemize}
\item finite set of variables $X = \{ x, y, z, \dots \}$
\item finite set of function symbols $F = \{ f, g, h, \dots \}$
\item finite set of predicate symbols $P = \{ p, q, r, \dots \}$
\end{itemize}
Each function symbol has a fixed ``arity'' (number of arguments), which can be zero (arity 0 gives a constant).
Each predicate symbol has also a fixed ``arity'' (number of arguments), which can be zero (arity 0 gives a proposition).

\subsubsection{Logical symbols}
\begin{itemize}
\item connectives ($\bot, \top, \neg, \wedge, \vee, \rightarrow$)
\item quantifiers ($\forall, \exists$)
\item finite set of predicate symbols $P = \{ p, q, r, \dots \}$
\end{itemize}
We don't add parentheses to the symbols; we will talk about syntax trees;
only if we want to write them down as strings, we add parentheses (as few as possible).

\subsubsection{Grammar}
\begin{itemize}
\item Terms: $\phi ~:=~ f_0 ~|~ f_n(t_1,\dots,t_n)$
\item Formulas: $\phi ~:=~ p_0 ~|~ p_n(t_1,\dots,t_n) ~|~ \bot ~|~ \top ~|~ \neg\phi ~|~ \phi\wedge\phi ~|~ \phi\vee\phi ~|~ \phi\rightarrow\phi ~|~ \forall x. \phi ~|~ \exists x. \phi$
\end{itemize}
When we write $\forall x. \phi$, every occurence of $x$ is bound in $\phi$.
A variable that is not bound is called free.
A change of bound variables does not change the abstract syntax tree (same in abstract syntax).

\subsubsection{Safe substitution}
$$\forall x\,.\;\, \exists y\,.\;\, y \ge x + 1$$
If we want to substitute $y^2$ for $x$, we must first rename $y$ to $z$ and then substitute.
$$\forall x\,.\;\, \exists z\,.\;\, z \ge x + 1$$
$$\exists z\,.\;\, z \ge y^2 + 1$$

}} % End localization of command definitions
