
\subsection{Decision Procedures}

\begin{remark}
    We now talk about three proof systems for validity: 
    \begin{align*}
            \overset{\text{Hilbert}}{\proofs \varphi} & \quad &  \overset{\text{Natural Deduction}}{\Gamma \proofs \varphi} & \quad &  
            \overset{\text{Grentzen}}{\Gamma \proofs \Delta}
        \end{align*}
\end{remark}
    
\begin{remark}
   We now talk abou three decision procedures (either for validity or for satisfiability)
\end{remark}


\begin{definition}
We define three decision procedures (either for validity or for satisfiability):
(i) Branching (if we don't derive $\bot$ then it is satisfiable)
\begin{center}
    \begin{prooftree}
    \AxiomC{$\Gamma[\top] \; \unsat$ }
    \AxiomC{$\Gamma[\bot]\; \unsat$}
    \BinaryInfC{$\Gamma[p]\; \unsat$}
\end{prooftree}
\end{center}
(ii) Resolution
\begin{center}
    \begin{prooftree}
    \AxiomC{$\Gamma, C_1, C_2, C_1[\bot] \lor C_2[\top]$ $\; \unsat$}
    \UnaryInfC{$\Gamma, C_1[p] , C_2[\neg p]$ $ \; \unsat$}
    \end{prooftree}
\end{center}
(iii) DPLL using unit resolution combined with branching of lower priority ($\ell$ is a literal)
\begin{center}
    \begin{prooftree}
    \AxiomC{$\Gamma, C[\bot]$ $\; \unsat$}
    \UnaryInfC{$\Gamma, \ell, C[\neg \ell]$ $ \; \unsat$}
    \end{prooftree}
\end{center}
\end{definition}

\begin{example}
Consider $p \lor q \lor r ,\; \neg p \lor \neg q \lor \neg r ,\; \neg p \lor q \lor r ,\; \neg q \lor r ,\; q \lor \neg r$
First we branch on $r$.
\begin{align*}
    \text{Case } r=\bot &: & p \lor q ,\; \neg p \lor q ,\; \neg q  &&\overset{Res. \neg q}{\longrightarrow} && p,\;\neg p && \overset{Res. \neg p}{\longrightarrow} &\bot \\
    \text{Case } r=\top &:  & \neg p \lor \neg q &&\overset{q=\top}{\longrightarrow} && \;\neg p && \overset{p=\bot}{\longrightarrow}& \bot
\end{align*}
\end{example}


\begin{definition}[Horn Clauses]
    Horn clause is a clause with at most one positive literal.
\end{definition}


\begin{remark}
    We can view Horn clauses as implications where LHS is a conjunction of positive propositions:
    \begin{align*}
\neg p \lor \neg q 
	&\iff p \land q \simplies \bot \\
\neg p \lor \neg q \lor r 
	&\iff p \land q \simplies r \\
r 
	&\iff \top \simplies r \\
\end{align*}
\end{remark}


\subsection{Metatheorems}

\begin{theorem}
    A countable set $\Gamma$ of formulas is satisfiable iff every finite subset of $\Gamma$ is satisfiable.
\end{theorem}

\begin{theorem}[Craig's interpolation]
    We have $\proofs \varphi \simplies \psi$ iff there exists a third formula $\chi$ (the "interpolant") which only uses
nonlogical symbols (in propositional logic, it is propositions only) that occur in
both $\varphi$ and $\psi$ such that $\proofs \varphi \simplies \chi$ and $\proofs \chi \simplies \psi$.
\end{theorem}

\begin{exercise}
    How hard is it to compute interpolates in propositional logic?
\end{exercise}

\begin{definition}
    The cut rule in $\nk$, $\nj$ (left) and $\lk$ (right) is respectively
    \begin{center}
        \AxiomC{$\Gamma \proofs \varphi$}
        \AxiomC{$\Gamma, \varphi \proofs \psi$}
        \BinaryInfC{$\Gamma \proofs \psi$}
        \DisplayProof
        $\quad$
        \AxiomC{$\Gamma \proofs \varphi, \Delta$}
        \AxiomC{$\Gamma, \varphi \proofs \Delta$}
        \BinaryInfC{$\Gamma \proofs \Delta$}
        \DisplayProof
    \end{center}
\end{definition}

\begin{theorem}[Cut elimination]
\label{cl8:thrm:cut}
    If a judgement can be proved with the cut rule, it can also be proved without the cut rule.
\end{theorem}

\begin{remark}[Theorem \ref{cl8:thrm:cut}]
In fact, it is ``iff''; the other direction is trivial.
Of course, the proof may become longer (introducing a lemma $\varphi$ often helps in practice).
It is a purely syntactic metatheorem. We cannot use deduction to prove it.
\end{remark}

% \subsubsection{Cut elimination}

% Cut rule in NK / NJ:
% $$\inferrule
% {\Gamma\proofs\varphi\qquad\qquad\qquad\Gamma,\varphi\proofs\psi}
% {\Gamma\proofs\psi}
% $$
% Cut rule in LK:
% $$\inferrule
% {\Gamma\proofs\varphi,\Delta\qquad\qquad\qquad\Gamma,\varphi\proofs\Delta}
% {\Gamma\proofs\Delta}
% $$

% If a judgement can be proved with the cut rule, it can also be proved without the cut rule.


\section{First-order logic}


\subsection{Definition}
\begin{remark}
    First-order logic, also called ``predicate logic'', is propositional logic with quantifiers $\forall$ (for all) and $\exists$ (exists).
\end{remark}

\begin{definition}[Signature]
    A Signature defines the non-logical symbols
\begin{itemize}
\item[(i)] finite set of variables $X = \{ x, y, z, \dots \}$
\item[(ii)] finite set of function symbols $F = \{ f, g, h, \dots \}$
\item[(iii)] finite set of predicate symbols $P = \{ p, q, r, \dots \}$
\end{itemize}
Each function symbol has a fixed ``arity'' (number of arguments), which can be zero (arity 0 gives a constant).
Each predicate symbol has also a fixed ``arity'' (number of arguments), which can be zero (arity 0 gives a proposition).
\end{definition}


\begin{definition}[Logical symbols]
The logical symbols consist of:
\begin{itemize}
\item[(i)] connectives ($\bot, \top, \neg, \land, \lor, \simplies$)
\item[(ii)] quantifiers ($\forall, \exists$)
\item[(iii)] finite set of predicate symbols $P = \{ p, q, r, \dots \}$
\end{itemize}
\end{definition}

\begin{remark}
We don't add parentheses to the symbols; we will talk about syntax trees;
only if we want to write them down as strings, we add parentheses (as few as possible).
\end{remark}

\begin{definition}[Syntax]
The syntax of first-order logic is given by the following grammar for terms and formulas repsectively
\begin{align*}
    t &\Coloneqq f_0 \; | \;  f_n(t_1,\dots,t_n) \\
    \varphi &\Coloneqq p_0  \; | \; p_n(t_1,\dots,t_n)  \; | \; \bot  \; | \; \top  \; | \; \neg\varphi  \; | \;\varphi\land\varphi  \; | \; \varphi\lor\varphi  \; | \; \varphi\simplies\varphi  \; | \; \forall x. \varphi  \; | \; \exists x. \varphi
\end{align*}
When we write $\forall x. \varphi$, every occurence of $x$ is bound in $\varphi$.
A variable that is not bound is called free.
A change of bound variables does not change the abstract syntax tree (same in abstract syntax).
\end{definition}


\begin{remark}[Safe substitution]
    If we want to substitute $y^2$ for $x$ in $\forall x\,.\;\, \exists y\,.\;\, y \ge x + 1$, we must first rename $y$ to $z$, i.e., $\forall x\,.\;\, \exists z\,.\;\, z \ge x + 1$, and then substitute, i.e., 
    $\exists z\,.\;\, z \ge y^2 + 1$.
\end{remark}

