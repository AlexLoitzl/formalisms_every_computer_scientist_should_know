\subsection{New types}


\begin{definition}
 To define simply typed $\lambda$-calculus we define:   types, terms, typing rules, reduction rules, and congruence rules
    \begin{itemize}
        \item The \emph{Types} are defined by the grammar
    \begin{align*}
        A \Coloneqq X \; | \; A_1\to A_2  \; | \; A_1 \times A_2 \; | \; 1
    \end{align*}
    where the type system of the simply typed lambda calculus is extended by the product type and unit type (empty product) respectively.
    \item The terms are defined by the following grammar
    \begin{align*}
       t \Coloneqq x \; | \; \lambda x \colon t \; | \; t_1, t_2 \; | \; \tuple{t_1, t_2} \; | \; \Pi_1(t) \; | \;  \Pi_2(t) \; | \; \tuple{}
    \end{align*}
    \item The \emph{typing rules} are,
    \begin{center}
        \AxiomC{}
        \UnaryInfC{$\Gamma \proofs \tuple{} \colon 1$}
        \DisplayProof
        $\quad$
        \AxiomC{$\Gamma \proofs t_1 \colon A$}
        \AxiomC{$\Gamma \proofs t_2 \colon B$}
        \BinaryInfC{$\Gamma \proofs \tuple{t_1, t_2} \colon A\times B$}
        \DisplayProof 
        $\quad$
        \AxiomC{$\Gamma \proofs t \colon A \times B$}
        \UnaryInfC{$\Gamma \proofs \pi_1(t) \colon A$}
        \noLine
        \UnaryInfC{$\Gamma \proofs \pi_2(t) \colon B$}
        \DisplayProof
    \end{center}
    \item The \emph{reduction rules} are,
    \begin{center}
        \AxiomC{}
        \UnaryInfC{$\pi_1(\tuple{t_1, t_2}) \to t_1$}
        \DisplayProof
        $\quad$
        \AxiomC{}
        \UnaryInfC{$\pi_1(\tuple{t_1, t_2}) \to t_2$}
        \DisplayProof
    \end{center}
 \item The\emph{congruence rules} are,
    \begin{center}
        \AxiomC{$s\to t$}
        \UnaryInfC{$\pi_1(s) \to \pi_1(t)$}
        \noLine
        \UnaryInfC{$\pi_2(s) \to \pi_2(t)$}
        \noLine
        \UnaryInfC{$\tuple{s,s'} \to \tuple{t,s'}$}
        \noLine
        \UnaryInfC{$\tuple{s',s} \to \tuple{s',t}$}
        \noLine
        \DisplayProof
    \end{center}
    \end{itemize}
    
   
\end{definition}
% $$A \colon = X|A \to A | A \times A \text{(product type)}| 1 \text{(unit type, empty product)}$$
% $$t \colon = x | \lambda x \colon  t | t_1 t_2 | <t_1, t_2> | \Pi_1(t) |  \Pi_2(t) |  <>$$


% \paragraph{Typing rules}
% $$\frac{\Gamma \proofs  t\colon A \times B }{\Gamma \proofs  \Pi_1(t)\colon A, \Gamma \proofs  \Pi_2(t)\colon B}$$
% $$\frac{\Gamma \proofs  t_1 \colon  A,  \Gamma \proofs  t_2 \colon  B  }{\Gamma \proofs  <t_1, t_2> \colon  A \times B}$$
% $$\frac{}{\Gamma \proofs  <> \colon  1}$$

% \paragraph{Reduction rules}
% $$\frac{}{\Pi_1(<t_1, t_2>) \to t_1}$$
% $$\frac{}{\Pi_2(<t_1, t_2>) \to t_2}$$

% \paragraph{Congruence rules}
% $$\frac{s \to t}{\Pi_1(s) \to \Pi_1(t), \Pi_2(s) \to \Pi_2(t), <s, s'> \to <t, s'>, <s', s> \to <s', t>}$$

\begin{remark}
    We will have:  (i) subject reduction, (ii) confluence, (iii) strong normalization, and (iv) curry-howard, i.e., $\to$ corresponds to $\simplies$, $\times$ corresponds to $\land$.
\end{remark}

\begin{remark}[Expressiveness over $\mathbb{N}$]
    Regarding the expressiveness over $\mathbb{N}$
    Simply typed $\lambda$-calculus (extended polynomial functions) can be enriched by (i) still staying total, e.g., type for primitive recursion (Godel "system T") and (ii) adding fixedpoints, e.g., $Y$.
    
\end{remark}


\begin{definition}[Plotkin's PCF]
Plotkin's Programming Computable Functions (PCF) expresses all computable functions (total and partial) while losing strong normalization.
They are defined w.r.t. to the grammar 
\begin{align*}
    A &\Coloneqq \NN \;|\; A_1 \to A_2 \\
    t &\Coloneqq  0 \;|\; Sx \;|\; \lambda x\colon A.t \;|\; t_1, t_2 \;|\; \mathrm{if }\; t_1 \; \mathrm{ then }\; t_2 \; \mathrm{ else }\; t3 \;|\; \mathrm{fix}_A
\end{align*}
and has the following typing rules
\begin{center}
    \AxiomC{}
    \UnaryInfC{$\Gamma \proofs  O \colon  \mathbb{N}$}
    \DisplayProof
    $\quad$
    \AxiomC{$\Gamma \proofs  x\colon \mathbb{N}$}
    \UnaryInfC{$\Gamma \proofs  Sx\colon \mathbb{N}$}
    \DisplayProof
    $\quad$
    \AxiomC{$\Gamma \proofs  t\colon  A \to A$}
    \UnaryInfC{$\Gamma \proofs  \mathrm{fix}_A t \colon  A$}
    \DisplayProof
\end{center}
\begin{center}
        \AxiomC{$\Gamma \proofs  t\colon \mathbb{N}$}
    \AxiomC{$\Gamma \proofs  s\colon A$}
    \AxiomC{$\Gamma \proofs  s'\colon A$}
    \TrinaryInfC{$\Gamma \proofs  \mathrm{if }\; t \; \mathrm{ then }\; s\; \mathrm{ else }\; s'  \colon  A$}
    \DisplayProof
\end{center}
and has the following reduction rules
\begin{center}
    \AxiomC{}
    \UnaryInfC{$\mathrm{if }\; 0 \; \mathrm{ then }\; s\; \mathrm{ else }\; s'  \to s$}
    \DisplayProof
    $\quad$
     \AxiomC{}
    \UnaryInfC{$\mathrm{if }\; Sx \; \mathrm{ then }\; s\; \mathrm{ else }\; s'  \to s'$}
    \DisplayProof
    $\quad$
      \AxiomC{}
    \UnaryInfC{$ \mathrm{fix}_A t \to t ( \mathrm{fix}_A t)$}
    \DisplayProof
\end{center}
\end{definition}
% \subsection{Plotkin's PCF ("programming computable functions")}
% Expresses all computable functions (total and partial) while losing strong normalizing.

% $$A\colon = N | A_1 \to A_2$$
% $$t \colon = O | Sx | \lambda x\colon A.t | t_1 t_2 | \text{if }t_1\text{ then }t_2\text{ else }t3 | fix_A$$

% $$\frac{}{ \Gamma \proofs  O \colon  \mathbb{N}}$$
% $$\frac{\Gamma \proofs  x\colon \mathbb{N} }{ \Gamma \proofs  Sx\colon \mathbb{N}}$$
% $$\frac{\Gamma \proofs  t\colon \mathbb{N},   \Gamma \proofs  s\colon A  , \Gamma\proofs  s'\colon A }{ \Gamma \proofs  \text{if }t\text{ then }s\text{ else }s'\colon A}$$
% $$\frac{\Gamma \proofs  t\colon  A \to A }{ \Gamma \proofs  fix_At\colon A}$$
% $$\frac{}{ fix_A t \to t (fix_A t)}$$


\begin{example}
    \label{cl14:ex:fact}
    The factorial function can be defined as follows: 
    \begin{align*}
        \mathrm{fact} \coloneqq \mathrm{fix}_{\mathbb{N} \to \mathbb{N}} (\lambda f. \lambda x. \;\mathrm{if }\;x\;\mathrm{ then }\;S0\;\mathrm{ else }\;x \times f(x-1))
    \end{align*}
\end{example}

\begin{exercise}
	Define multiplication and subtraction in Example \ref{cl14:ex:fact}.
\end{exercise}

\subsection{Typed $\lambda$ calculus "without types"}


\begin{remark}
    Type inference is concerned with checking typability, i.e., i.e., given closed $t$, is there a type $A$ such that $\proofs  t\colon A$?, if so, find $A$ . For typed $\lambda$-calculus "without types" this can be checked in linear time. 
\end{remark}


\begin{definition}
    Typed $\lambda$ calculus "without types" follows the grammar
    \begin{align*}
        A \Coloneqq X \;|\; A_1 \to A_2 \;|\; 1
        t \Coloneqq  \;|\; \lambda x.t \;|\; t_1 t_2
    \end{align*}
    and uses the following typing rules
    \begin{center}
        \AxiomC{}
        \UnaryInfC{$ \Gamma \proofs  x \colon \Gamma(x)$}
        \DisplayProof
        $\quad$
        \AxiomC{$\Gamma, x\colon A \proofs  t\colon B$ }
        \UnaryInfC{$  \Gamma \proofs  \lambda x.t\colon A \to B$}
        \DisplayProof
        $\quad$
        \AxiomC{$\Gamma \proofs  t \colon A \to B$}
        \AxiomC{$ \Gamma \proofs  s\colon A $}
        \BinaryInfC{$ \Gamma \proofs  ts \colon B$}
        \DisplayProof
        $\quad$
    \end{center}
\end{definition}


\begin{remark}
    In this type system terms can have multiple types, e.g., 
    \begin{align*}
        \lambda x.x \colon 1 \to 1 \quad \text{and} \quad \lambda x.x \colon (1 \to 1) \to 1 \to 1
    \end{align*}
    However, there is always a \emph{most general} type called the principal type, e.g., $\lambda x.x \colon  X \to X$.
\end{remark}


\begin{remark}[Substitution on types]
    We denote $\sigma\coloneqq A[X \mapsto B]$ substitution on types.
    If $\proofs  t\colon A$ then for all substitutions $\sigma$, $\proofs  t \colon  A\sigma$.
\end{remark}
   

\begin{theorem}[Principal type]
\label{cl14:thrm:principal}
	If $\proofs  t\colon A$, then there exists $B$ such that for all $B'$ is $\proofs  t \colon  B'$ then exists $\sigma$ such that $B' = B\sigma$.
\end{theorem}



% $$\frac{}{}$$
% $$\frac{\Gamma, x\colon A \proofs  t\colon B }{ \Gamma \proofs  \lambda x.t\colon A \to B}$$
% $$\frac{\Gamma \proofs  t \colon A \to B , \Gamma \proofs  s\colon A }{ \Gamma \proofs  ts \colon B}$$



% Note: Terms can have multiple types. e.g. 
% $$\lambda x.x \colon 1 \to 1$$  $$\lambda x.x \colon (1 \to 1) \to 1 \to 1$$
\begin{remark}[cont.~Thrm \ref{cl14:thrm:principal}]
    To find $B$, collect ``$=$''-constraints on types. In particular, type variables are equal, i.e. $A = A"$, if they can be unified, i.e., $\exists \sigma, \sigma'\colon  A\sigma = A'\sigma'$.
    Judgments can be used to add ``$=$''-constraints on types to typing rules, i.e, $\Gamma \proofs  t\colon A \; |\;  E $ where $E$ is a set of equations on types and for $Y$ being free
    \begin{prooftree}
        \AxiomC{$\Gamma \proofs  t \colon  A | E$}
        \AxiomC{$\Gamma \proofs  s \colon  B | F$}
        \BinaryInfC{$\Gamma \proofs  ts    |     E \cup F \cup \{A = B \to Y\}$}
    \end{prooftree}
    
This gives you an algorithm for computing principal types.
\end{remark}


\begin{definition}[Polymorphic types]
    Polymorphic types, i.e., $\lambda x.x \colon  \forall X.X\to X$, are the simplest form is Hinolley-Milner types. 
    They follow the grammar
    \begin{align*}
        A \Coloneqq X \; |\; A \to A \; |\; \forall X.A
    \end{align*}
\end{definition}



% \subsection{IMP}

% \begin{definition}[IMP Syntax]
%     IMP is a simple imperative language corresponding to PCF. Let $X$ be a set of variables, intuitively they are locations, then we can define IMP syntactically with the following grammar
%     \begin{align*}
%         Aexp&\Coloneqq  a \coloneqq 0 \;|\; X \;|\; Sa \\
%         Cexp&\Coloneqq  c \coloneqq \sskip \;|\; X\coloneqq a \;|\; c_1 ; c_2 \;|\; 
%         \sif{a}{c_1}{c_2}\;|\; \swhile{a}{c}.
%     \end{align*}
% \end{definition}

% \begin{remark}
%     There are two semantis, i.e., the operational semantics (which evaluates the expression) and the denotational semantics.
% \end{remark}

% \begin{definition}[IMP Semantics]
% Let $X$ be a setof variables representing locations.
% A state $\sigma\colon  X \to \mathbb{N}$ is a function from locations to $\NN$. We denote $\Sigma$ as the set of states, then
% $\eval{a}\colon \Sigma\to \NN$ and $\eval{c}\colon \Sigma \to \Sigma$. 
% The function $\eval{c}$ is a partial function, thus we can alternatively denote is as  $\eval{c}\colon \Sigma \to \Sigma_{\bot}$ where $\Sigma_{\bot}\coloneqq \Sigma \cup \{\bot\}$ with $\bot$ representing non-termination.     
% \end{definition}




