{{%Localize command definitions

\chapter{Class 6}
% ChatGPT was used when working on this transcript.

Formal system $F$ is a set of rules.
Rule is a finite set of (formulas) premises $p_0, \dots, p_k$ and (a formula called) conclusion $c$.
We usually have infinitely many rules but only finitely many different rule schemata.
For example, schema $\phi \rightarrow \phi$ gives infinitely many rules like $p_3 \rightarrow p_3$.
Axiom is a rule without premises.\bigskip\\
Proof (derivation) is a finite sequence of formulas $\phi_0, \dots, \phi_n$ such that every formula in the sequence is
\begin{itemize}
\item either an axiom (which can be viewed as a special case of the following);
\item or the conclusion of a rule whose premises occur earlier in the sequence.
\end{itemize}
This is a linear view.

Linear view is usually easier for proving meta theorems.
Tree view (inductive definition) is usually better in practice.

Theorem is a formula that occurs in a proof. We distinguish the following:
\begin{itemize}
\item $\vdash \phi \;\dots\;$ ``$\phi$ is a theorem (of the formal system $F$)'' (has a proof) [syntax]
\item $\vDash \phi \;\dots\;$ ``$\phi$ is valid ($\phi$ is tautology)'' (is true in all models) [semantics]
\end{itemize}
Formal system equipped with semantics is called a logic.
Most of logic is about establishing $\vdash \phi$ iff $\vDash \phi$.

Rule $R$ is sound iff [if all premises of $R$ are valid, then the conclusion of $R$ is valid].
Formal system $F$ is sound iff all rules are sound (or equivalently, every theorem is valid).
Formal system $F$ is complete iff every valid formula is a theorem.
Formal system $F$ is consistent unless $\vdash \bot$ (or equivalently, there exists a formula that is not a theorem).
Rule $R$ is derivable in $F$ iff [for all formulas $\phi$, $\underset{F \cup \{R\}}\vdash \phi$ iff $\underset{F}\vdash \phi$].
Rule $R$ is admissible in $F$ iff $F \cup \{R\}$ is still consistent.
Formula $\phi$ is expressible in a logic $L$ iff [there exists a formula $\psi$ of $L$ such that, for all interpretations $v$, $[[ \phi ]]_v = [[ \psi ]]_v$. %`⟦φ⟧ᵥ = ⟦ψ⟧ᵥ`].
For example $\phi_1 \wedge \phi_2$ is expressible using only $\neg$ and $\vee$ (de Morgan)
as $\psi = \neg(\neg\phi_1 \wedge \neg\phi_2)$.

We can enumerate all theorems by systematically enumerating all possible proofs.
The proof is a witness for validity.
Sound formal system gives a sound procedure for validity (but not necessarily complete).
Sound complete formal system gives a sound semi-complete procedure for validity (may not terminate on inputs
that represent a formula that is not valid).
To get a decision procedure (sound and complete procedure for validity), we need both
(1) sound complete formal system for validity, and
(2) sound complete formal system for satisfiability (to define a formal system for satisfiability,
replace ``formulas'' ($\phi$ is valid) by ``judgements'' ($\phi$ is satisfiable);
all axioms are satisfiable, all rules go from satisfiables to satisfiable).
For every input $\phi$, one of them will eventually terminate.
Conclude; either $\phi$ is valid, or $\neg\phi$ is satisfiable (which means that $\phi$ is not valid).
Recall that, if both a set and its complement are recursively-enumerable, the set is recursive (decidable).

Example (formal system for unsatisfiability):
$$\inferrule
{\Gamma[\bot]\qquad\qquad\Gamma[\top]}
{\Gamma[p]}
$$

\section{Hilbert formal system for propositional logic}

Hilbert system uses connectives $\rightarrow$ and $\neg$ only.
Hilbert system has three axioms and one rule -- modus ponens (MP):
$$\inferrule
{\phi\qquad\qquad\qquad\phi\rightarrow\psi}
{\psi}
$$
Axioms:
\begin{itemize}
\item (K): $\qquad
\phi\rightarrow\psi\rightarrow\phi$
\item (S): $\qquad
(\phi\rightarrow\psi\rightarrow\chi)\rightarrow((\phi\rightarrow\psi)\rightarrow(\phi\rightarrow\chi))$
\item (EM): $\qquad
(\neg\phi\rightarrow\neg\psi)\rightarrow(\psi\rightarrow\phi)$
\end{itemize}
TODO

}} % End localization of command definitions
