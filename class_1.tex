
% \section{Tools}
% \begin{itemize}
%   \item Lean or Coq 
%   \item CVC5 or Z3 
%   \item possibly a model checker 
%   \item professional version of ChatGPT
% \end{itemize}

% \section{Syllabus}
% \begin{enumerate}
%   \item MATH (``Informalism'')
%   \begin{itemize}
%     \item proofs (natural deduction)
%     \item fixpoints (induction, coinduction)
%   \end{itemize}
%   \item DECLARATIVE LOGIC
%   \begin{itemize}
%     \item syntax (rules) vs. semantics (models)
%     \item propositional, predicate, modal logic 
%     \item decision procedures (SAT, SMT)
%   \end{itemize}
%   \item FUNCTIONS
%   \begin{itemize}
%     \item $\lambda$ calculus, typed $\lambda$
%     \item SOS (structured operational semantics), rewriting
%     \item ``propositions-as-types'' (connection to logic)
%   \end{itemize}
%   \item PROCESSES (CONCURRENT)* 
%   \begin{itemize}
%     \item CCS, Petri nets
%     \item (bi-) simulation
%   \end{itemize}
%   \item CIRCUITS*
%   \begin{itemize}
%     \item boolean, sequential, dataflow (Kahn nets)
%     \item interfaces
%   \end{itemize}
%   \item STATE TRANSITION SYSTEMS*
%   \begin{itemize}
%     \item ($\omega$-) automata, games, timed, probabilistic, pushdown
%     \item programs, Turing machines 
%     \item grammars (Chomsky hierarchy)
%   \end{itemize} 
%   \item DECLARATIVE SPECIFICATION
%   \begin{itemize}
%     \item Hoare logics, separation logic 
%     \item temporal logics (LTL, CTL, ATL)
%     \item partial correctness vs. termination, safety vs. liveness 
%   \end{itemize}
% \end{enumerate}

% *: Operational.

\section{Lecture 1: How to write an informal proof}

\subsection{Example 1: Sum of two bounded functions}

\begin{definition}
  A real number $b \in \RN$ is a \emph{bound} of a 
  function $f\colon \RN\to \RN$ from $\RN$ to $\RN$, if for all $x$ in 
  $\RN$, we have $f(x) \leq b$. 
\end{definition}


\begin{definition}
  Given two functions $f\colon  \RN\to \RN$ and $g\colon  \RN\to \RN$ from $\RN$ to $\RN$, their \emph{sum} is the function $f + g$ such 
  that for all $x$ in $\RN$, we have $(f+g) (x) = f(x) + g(x)$.
\end{definition} 


\begin{theorem}
  For all functions $f\colon  \RN\to \RN$ and $g\colon  \RN\to \RN$, if $f$ and
  $g$ are bounded, then $f+g$ is bounded. 
\end{theorem}
\begin{proof}
Consider arbitrary functions $\hat{f}$ and $\hat{g}$ from $\RN$ to $\RN$.
We assume that $\hat{f}$ and $\hat{g}$ are bounded, and want to 
show that $\hat{f} + \hat{g}$ is bounded.

By the definition of boundedness, we show that there exists a $b \in \RN$ such
that for all $x \in \RN$, $(\hat{f} + \hat{g})(x) \leq b$.

For this, let $\hat{b_1}$ be such that for all $x \in \RN$, $\hat{f}(x) \leq \hat{b_1}$.
Similarly, let $\hat{b_2}$ be a bound for $\hat{g}$.
We show that $\hat{b_1} + \hat{b_2}$ is a bound for $\hat{f} + \hat{g}$, i.e.
for all $x \in \RN$, we have $(\hat{f} + \hat{g})(x) \leq \hat{b_1} + \hat{b_2}$.

Consider an arbitrary $\hat{x} \in \RN$. We show that $(\hat{f} + \hat{g})(\hat{x}) \leq \hat{b_1} + \hat{b_2}$.
By the definition of + for functions, we have to show $\hat{f}(\hat{x}) + \hat{g}(\hat{x}) \leq \hat{b_1} + \hat{b_2}$.

For all $x_1, x_2, y_1, y_2 \in \RN$, if $x_1 \leq y_1$ and $x_2 \leq y_2$, then $x_1 + x_2 \leq y_1 + y_2$. 
This can be broken down even further:

\begin{enumerate}
  \item[(i)] for all $x,y,z \in \RN$, if $x \leq y$ and $y \geq z$, then $x \leq z$
  \item[(ii)] for all $x,y \in \RN$, $x + y = y + x$
  \item[(iii)] for all $x,y,z \in \RN$, if $x \leq y$, then $x + z \geq y + z$, then $x \leq z$
\end{enumerate}
\end{proof}

\subsection{Example 2: Schröder-Bernstein Theorem}

\begin{definition}
  Two sets $A$ and $B$ are \emph{equipollent} (``have the same 
  size''), if there is a bijection from $A$ to $B$.
\end{definition}

\begin{definition}
  A function $f$ from $A$ to $B$ is: 
  (i) \emph{one-to-one} if for all $x$ and $y$ in $A$, if $x \neq y$, then $f(x) \neq f(y)$;
  (ii) \emph{onto} if for all $z$ in $B$, there exists $x$ 
    in $A$ such that $f(x) = z$;
(iii) \emph{bijective} if $f$ is one-to-one and onto.
\end{definition}




% \begin{center}
% \begin{tabular}{
%   p{0.35\linewidth} | p{0.35\linewidth} | p{0.2\linewidth}}
%   Goals & Knowledge & Outermost symbol \\
%   \hline
%   \begin{itemize}[leftmargin=*]
%     \item[] Show for all $x$, $G(x)$.
%     \item[] Consider arbitrary $\hat{x}$.
%     \item[] Show $G(\hat{x})$. 
%   \end{itemize} & \begin{itemize}[leftmargin=*]
%     \item[] We know for all $x$, $K(x)$.
%     \item[] In particular, we know $K(\hat{t})$.
%     \item[] $\hat{t}$: term containing only constants. 
%   \end{itemize} & \begin{itemize}
%     \item[] \Large $\forall$
%   \end{itemize} \\ 
%   \hline 
%   \begin{itemize}[leftmargin=*]
%     \item[] Show there exists $x$ s.t. $G(x)$.
%     \item[] We show that $G(\hat{t})$.
%     \item[] $\hat{t}$: term containing only constants.
%   \end{itemize} & \begin{itemize}[leftmargin=*]
%     \item[] We know there exists $x$ s.t. $K(x)$.
%     \item[] Let $\hat{x}$ be s.t. $K(\hat{x})$.
%   \end{itemize} & \begin{itemize}
%     \item[] \Large $\exists$
%   \end{itemize}
% \end{tabular}
% \end{center}

\begin{theorem}[Schr\"{o}der-Bernstein theorem]
    \label{thrm:class1:bernstein}
    If there is a one-to-one function on a set $A$ to a subset of a set $B$ and
    there is also a one-to-one function on $B$ to a subset of $A$, then $A$ and
    $B$ are equipollent.
\end{theorem}

\begin{exercise}
    Correct the following incorrect proof of Theorem \ref{thrm:class1:bernstein} \emph{in the presented style}:
    \begin{quote}
Suppose that $f$ is a one-to-one map of $A$ into $B$ and $g$ is
one-to-one on $B$ to $A$. It may be supposed that $A$ and $B$ are disjoint. The proof of the theorem is accomplished by decomposing $A$ and $B$ into classes which are most easily described in terms of parthenogenesis. A point $x$ (of either $A$ or $B$) is an ancestor of a point $y$ iff $y$ can be obtained from $x$ by successive application of $f$ and $g$ (or $g$ and $f$). Now decompose $A$ into three sets:
let $A_e$ consist of all points of $A$ which have an even number of ancestors, let $A_o$ consist of points which have an odd number of ancestors, and let $A_{\infty}$ consist of points with infinitely many ancestors. Decompose $B$ similarly and observe: f maps $A_e$ onto
$B_o$ and $A_{\infty}$ onto  $B_{\infty}$, and $g^{-1}$ maps $A_o$ onto $B_e$. Hence the function which agrees with $f$ on $A_e\cup A_{\infty}$, and agrees with $g^{-1}$ on $A_o$ is a one-to-one map of $A$ onto $B$.
    \end{quote}
\end{exercise}

\newpage
