
\begin{remark}
    To summarize:
    \begin{align*}
        \equiso &&  \mathrm{NP} && -\\
        \equbis && O(m log n) && \HML\\
        \equsim && O(m \cdot n) && \HML^{\exists}\\
         \equlin && \mathrm{PSPACE} && \HMLlin^{\exists}\\
    \end{align*}
\end{remark}
% \begin{itemize}
% 	\item $\approx_{iso} \to NP \to HML$.
% 	\item $\approx_{bis} \to O(m log n) \to HML^\exists$. (gfp)
% 	\item $\approx_{sim} \to O(m n) \to HML_{lin}^\exists$. (gfp)
% 	\item $\approx_{lin} \to pspace$.
% 	\item $[a](\varphi \land \psi) = [a]\varphi \land [a]\psi$
% 	\item $<a>(\varphi \lor \psi) = <a>\varphi \lor <a>\psi$
% 	\item $[\{a, b\}] \varphi = [a] \varphi \land [b] \varphi$
% 	\item $<\{a, b\}> \varphi = <a> \varphi \lor <b> \varphi$
% \end{itemize}

\section{Modal $\lmu$-calculus}

\begin{definition}[Modal $\lmu$-calculus Syntax]
    The syntax of the modal $\lmu$-calculus is defined by the following grammar. Let $X$ be a set of variables
    \begin{align*}
        \varphi \Coloneqq \HML \; \mid \;  x \in X \; \mid \; \underbrace{ \lmu x. \varphi}_{\lfp} \; \mid \;  \underbrace{\lnu x. \varphi}_{\gfp}
    \end{align*}
\end{definition}

\begin{definition}[Modal $\lmu$-calculus Semantics]
    The semantics of the modal $\lmu$-calculus is defined as follows.
    Let $T\coloneqq (S,\to)$ be a LTS over a finite action set (alphabet) $A$.
    The we define: (i) $ \eval{\varphi}_T^\interp \subseteq S$ where $\interp \colon X \to 2^S$; 
    (ii) $\eval{x}_T^\interp  = \interp(x)$; 
    (iii)  $\eval{\lmu x. \varphi}_T^\interp  = \bigcup_{\sigma \subseteq S} \eval{\varphi}_T^{\interp[x \to \sigma]}$; (iv) $\eval{\lnu x. \varphi}_T^\interp  = \bigcap_{\sigma \subseteq S} \eval{\varphi}_T^{\interp[x \to \sigma]}$.
\end{definition}

\begin{remark}
        Let $F\coloneqq \lambda \sigma \subseteq S.\eval{\varphi}_T^{\interp[x \to \sigma]}$ then
    $\eval{\lmu x. \varphi}_T^\interp = \lfp F$ and $\eval{\lnu x. \varphi}_T^\interp = \gfp F$.
    That is, $\lfp F$ is the limit (infinite union) of 
    \begin{align*}
        \sigma_0 = \emptyset 
        \subseteq \sigma_1 = \eval{\varphi}_T^{\interp[x \to \sigma_0]}
        \subseteq  \sigma_2 = \eval{\varphi}_T^{\interp[x \to \sigma_1]} 
        \subseteq \cdots
    \end{align*}
    and $\lfp F$ is the limit (infinite intersection) of 
    \begin{align*}
        \sigma_0 = S
        \supseteq \sigma_1 = \eval{\varphi}_T^{\interp[x \to \sigma_0]}
        \supseteq  \sigma_2 = \eval{\varphi}_T^{\interp[x \to \sigma_1]} 
        \supseteq \cdots
    \end{align*}
\end{remark}

\begin{example}
    Consider the expression $(\lmu x)(\bev{a} \true \lor \bev{\bar{a}}x)$ where $\bar{a}\coloneqq A\setminus \{a\}$. By evaluating the expression we obtain
    \begin{align*}
        \sigma_0 &= \false \\
        \sigma_1 &= \bev{a}\true \\
        \sigma_2 &= \bev{a}\true  \lor \bev{\bar{a}}\bev{a}\true \\
        \sigma_3 &= \bev{a}\true  \lor \bev{\bar{a}} (\bev{a}\true  \lor \bev{\bar{a}}\bev{a}\true)\\
        &= \bev{a}\true  \lor \bev{\bar{a}}\bev{a}\true \lor \bev{\bar{a}}\bev{\bar{a}}\bev{a}\true\\
        \sigma_4 &= \cdots
    \end{align*}
    It expresses that 
    there exists a path somewhere along the path there exists an $a$-transition, i.e., $\exists \lev a$. 
\end{example}

\begin{example}
    Consider the expression $(\lnu x)(\bal{\bar{a}} \false \land \bev{a}x)$. By evaluating the expression we obtain
    \begin{align*}
        \sigma_0 &= \true \\
        \sigma_1 &= \bal{\bar{a}}\false \\
        \sigma_2 &= \bal{\bar{a}}\false \land \bal{a}\bal{\bar{a}}\false \\
        \sigma_3 &= \cdots
    \end{align*}
    It expresses that 
    on all paths there are only $a$-transition, i.e., $\forall \lal a$. 
\end{example}

\begin{remark}
 The the equivalence $\neg \lmu x. \varphi(x) = \lnu x . \neg \varphi (\neg x)$ where left has even number of negations.
\end{remark}



\begin{definition}[$\lmu$-calculus]
We can add propositions $p\in P$ as atomic formulas. Syntactically we can define the grammar
 \begin{align*}
        \varphi \Coloneqq p \; \mid \; \varphi \lor \varphi\; \mid \;  \varphi \land \varphi \; \mid \;  \bev{a}\varphi  \; \mid \; \bal{a}\varphi  \; \mid \;  \lmu x. \varphi \; \mid \; \lnu x. \varphi x
\end{align*}
    Semantically we evaluate over Kripke structures, i.e., $K=(S, \to, \eval{\cdot}\colon S\to 2^P)$, e.g., $\eval{p}_K\subseteq S$
\end{definition}

\begin{remark}
    We can formulate the following abbreviations:
    \begin{align*}
        \exists \lev \varphi &= \lmu x. (\varphi \lor \exists \lne x) \\
        \exists \lev p &= \lmu x. (p \lor \exists \lne x) & \quad \text{($p$ is reachable)} \\
         \forall \lal \varphi &= \lnu x. (\varphi \land \forall \lne x) \\
         \forall \lal p &= \lnu x. (p \land \forall \lne x)  & \quad \text{($p$ is an invariant)}  \\
         \forall \lev \varphi &= \lmu x. (\varphi \lor \forall \lne x) & \quad \text{(on all paths $\varphi$ is eventually true)} \\
          \exists \lev \varphi &= \lnu x. (\varphi \land \exists \lne x) & \quad \text{(there exists an inf.\ path where $\varphi$ is always true)} 
    \end{align*}
\end{remark}


% 	\item
% 	\item $\lmu$-calculus add propositions as atomic formulas i.e. $\varphi := p | \varphi \lor \varphi | \varphi \land \varphi | <a> \varphi | [a] \varphi | \lmu x.\varphi | \lnu x.\varphi | X$
% 	\item $\llbracket p\rrbracket_T \subset S$
% 	\item LTS $T = (S, \to) $ Kripke structure $K = (S, \to, \llbracket\rrbracket:S \to 2^P)$
% \end{itemize}

% Abbreviations:
% \begin{itemize}
% 	\item $\exists \lev \varphi = \lmu x. (\varphi \lor \exists \lne x)$
% 	\item $\exists \lev p = \lmu x. (p \lor \exists \lne x)$ i.e. p is reachable.
% 	\item $\forall \lal \varphi = \lnu x. (\varphi \land \forall \lne x)$
% 	\item $\forall \lal p = \lnu x. (p \land \forall \lne x)$ i.e. p is an invariant.
% 	\item $\forall \lev \varphi = \lmu x. (\varphi \lor \forall \lne x)$
% 	\item $\exists \lal \varphi = \lnu x. (\varphi \land \exists \lne x)$ i.e. there exists an infinite path along which $\varphi$ is always true.
% \end{itemize}
% equivalant to
% $$\llbracket \lmu x \varphi \rrbracket_T^I = \text{lfp}(F)$$
% for 
% $$F = \lambda \sigma \subset S. \llbracket\varphi \rrbracket_T^{I[x \to \sigma]}$$
% $\text{lfp}(F)$ is the limit (inf union) of:
% $$\sigma_0 = \varnothing$$
% $$\subseteq \sigma_1 = \llbracket\varphi \rrbracket_T^{I[x \to \sigma_0]}$$
% $$\subseteq \sigma_2 = \llbracket\varphi \rrbracket_T^{I[x \to \sigma_1]}$$
% $$\dots$$
% $$\subseteq S$$

% And same for $\lnu$:
% $$\llbracket \lnu x \varphi \rrbracket_T^I = \cap_{\sigma \subset S} \llbracket\varphi \rrbracket_T^{I[x \to \sigma]}$$
% equivalant to
% $$\llbracket \lnu x \varphi \rrbracket_T^I = \text{gfp}(F)$$
% for 
% $$F = \lambda \sigma \subset S. \llbracket\varphi \rrbracket_T^{I[x \to \sigma]}$$
% $\text{gfp}(F)$ is the limit (inf intersection) of:
% $$\sigma_0 = S$$
% $$\supseteq \sigma_1 = \llbracket\varphi \rrbracket_T^{I[x \to \sigma_0]}$$
% $$\supseteq \sigma_2 = \llbracket\varphi \rrbracket_T^{I[x \to \sigma_1]}$$
% $$\dots$$


% \begin{example}
% $$\lmu x . (<a> true \lor <\bar{a}> x)$$
% solution:
% $$\sigma_0 = false$$
% $$\sigma_1 = <a> true$$
% $$\sigma_2 = <a> true \lor <\bar{ a}><a>true$$
% $$\sigma_3 = <a> true \lor <\bar{ a}>(<a>true \lor <\bar{ a}><a>true)$$
% $$= <a> true \lor <\bar{ a}><a>true \lor <\bar{ a}><\bar{ a}><a>true$$
% $$\sigma_4 = \dots$$

% means "$\exists \lev a$"
% there exists a path somewhere along the path there exists an a transition.

% \end{example}

% \begin{example}
% $$(\lnu x)([\bar{ a}] flase \land [a] x)$$
% solution:
% $$\sigma_0 = true$$
% $$\sigma_1 = [\bar{ a}]false \land [a]true$$
% $$\sigma_2 = [\bar{ a}]false \land [a][\bar{ a}]false$$
% $$\sigma_3 = \dots$$

% means  "$\forall \lal a$" 
% on all paths only a transitions.
% \end{example}

% Some notes:
% \begin{itemize}
% 	\item $\neg \lmu x. \varphi(x) = \lnu x . \neg \varphi (\neg x)$ where left has even number of negations.
% 	\item $\lmu$-calculus add propositions as atomic formulas i.e. $\varphi := p | \varphi \lor \varphi | \varphi \land \varphi | <a> \varphi | [a] \varphi | \lmu x.\varphi | \lnu x.\varphi | X$
% 	\item $\llbracket p\rrbracket_T \subset S$
% 	\item LTS $T = (S, \to) $ Kripke structure $K = (S, \to, \llbracket\rrbracket:S \to 2^P)$
% \end{itemize}

% Abbreviations:
% \begin{itemize}
% 	\item $\exists \lev \varphi = \lmu x. (\varphi \lor \exists \lne x)$
% 	\item $\exists \lev p = \lmu x. (p \lor \exists \lne x)$ i.e. p is reachable.
% 	\item $\forall \lal \varphi = \lnu x. (\varphi \land \forall \lne x)$
% 	\item $\forall \lal p = \lnu x. (p \land \forall \lne x)$ i.e. p is an invariant.
% 	\item $\forall \lev \varphi = \lmu x. (\varphi \lor \forall \lne x)$
% 	\item $\exists \lal \varphi = \lnu x. (\varphi \land \exists \lne x)$ i.e. there exists an infinite path along which $\varphi$ is always true.
% \end{itemize}

\subsection{Computational Tree Logic (CTL)}

\begin{definition}[CTL Syntax]
    The syntax of the Computation Tree Logic (CTL) is given by the following grammar
    \begin{align*}
        \varphi \Coloneqq p \;\mid\; 
        \varphi \lor \varphi \;\mid\; 
        \varphi \land \varphi \;\mid\; 
        \overbrace{\exists \lne \varphi}^{EX} \;\mid\; 
        \overbrace{\forall \lne \varphi}^{AX} \;\mid\;
        \overbrace{\exists \lev \varphi}^{EF} \;\mid\;
        \overbrace{\forall \lev \varphi}^{AF} \;\mid\; 
        \overbrace{\exists \lal \varphi}^{EG} \;\mid\;
        \overbrace{\forall \lal \varphi}^{AG}
    \end{align*}
\end{definition}

\begin{remark}
    The following equivalences hold: $\exists \lne \varphi = \neg \forall \lne \neg \varphi$ and $\exists \lev \varphi = \neg \forall \lal \neg \varphi$.
\end{remark}

\begin{remark}
    The full $\lmu$-calculus is strictly more expressive than CTL, in two ways: (i) $\lmu x.\exists \lne \exists \lne x$ means $p$ is reachable in even number of steps. (ii) $\lnu x. \exists \lev (p \land \exists \lne x) = \lnu x .\, \lmu y ((p \land \exists \lne x) \lor \exists \lne y)$ means there exists a path with infinitely many $p$ clauses. It is not expressible in CTL.
\end{remark}

\begin{exercise}
    Proof that $\lmu x.\exists \lne \exists \lne x$ is not expressible in CTL.
\end{exercise}


\begin{remark}
    We have that $\forall \lal \forall \lev p = \forall \lal \lev p$ expresses that on all infinite paths, we have $p$ infinitely often. 
     By contrast, while
     $\exists \lal \exists \lev p \Leftarrow  \exists \lal \lev p$ the reverse does not hold, e.g.,
     \begin{center}
		\begin{tikzpicture}
			\node (S) at (10,11) {};
			\node[circle,draw, minimum size= 0.7cm, inner sep=1.5pt] (A) at (10,10) {\small$\neg p$};
			\node[circle,draw, minimum size= 0.7cm, inner sep=1.5pt] (B) at (12,10) {\small$p$};
			\path[->]
			(S) edge (A)
			(A) edge (B)
			(A) edge [out=135, in=225, loop] (A);
			%\draw[->] (w0) -- (w1);
		\end{tikzpicture}
	\end{center}
 Moreover, $CTL$ characterizes $\equbis$ and  $CTL^\exists$ characterizes $\equsim$.
 Next we define a logic that characterizes $\equlin$.
\end{remark}



\subsection{Linear Temporal Logic (LTL)}

\begin{remark}
    LTL characterizes $\equlin$.
\end{remark}


\begin{definition}[LTL Snytax]
    The syntax of Linear Temporal Logic (LTL) is given by the following grammar
    \begin{align*}
        phi \Coloneqq p \;\mid\;  
        \varphi \lor \varphi \;\mid\; 
        \varphi \land \varphi \;\mid\; 
        \lne \varphi \;\mid\;  
        \lev \varphi 
        \;\mid\; \lal \varphi
    \end{align*}
\end{definition}

\begin{definition}[LTL Semantics]
LTL is interpreted over inf traces $t = s_0s_1s_2s_3\dots$ s.t $s_i \to s_{i+1}$ for all $i \geq 0$ the we define 
\begin{align*} 
&t \models p &\iff \quad&p \in \eval{s_0} & \\
&t \models \lne \varphi &\iff \quad&s_0s_1s_2s_3\dots \models \varphi & \quad \text{next $\varphi$}\\
&t \models \lev \varphi &\iff \quad&\exists i\geq0 s_is_{i+1}s_{i+2}s_{i+3}\dots \models \varphi & \quad \text{eventually $\varphi$}\\
&t \models \lal \varphi &\iff \quad&\forall i\geq0 s_is_{i+1}s_{i+2}s_{i+3}\dots \models \varphi & \quad \text{always $\varphi$}\\
\end{align*}
We say $s \models \varphi$ iff there exists trace $t$ from $s$ s.t. $t \models \varphi$.
\end{definition}

\begin{remark}
    Now you can compare LTL to CLT etc. For example $\lal \lev p$ is infinitely often $p$. It is incomparable to CTL (fragment of of $\lmu$-calculus).
% This logic was defined as a first order logic, but we should also add until operator to it to be complete with respect to first order logic.
\end{remark}

\begin{definition}[Until]
We define the until operator $\lun$ as 
\begin{align*}
    \varphi\lun\psi \iff \exists s_is_{i+1}\dots \models \psi \land \forall j<i. s_js_{j+1}\dots\models \varphi
\end{align*}
\end{definition}

% \begin{itemize}
% 	\item $t \models p$ iff $p \in \llbracket s_0\rrbracket$
% 	\item $t \models \lne \varphi$ iff $s_1s_2s_3\dots \models \varphi$. next
% 	\item $t \models \lev \varphi$ iff $\exists i\geq0 s_is_{i+1}s_{i+2}s_{i+3}\dots \models \varphi$. eventually
% 	\item $t \models \lal \varphi$ iff $\forall i\geq0 s_is_{i+1}s_{i+2}s_{i+3}\dots \models \varphi$ always
% \end{itemize}
% We say $s \models \varphi$ iff there exists trace $t$ from $s$ s.t. $t \models \varphi$.


% Now you can compare LTL to CLT etc. For example $\lal \lev p$ is inf often p. It is not comparable to CTL, fragment of of $\lmu$-calculus.

% This logic was defined as a first order logic, but we should also add until operator to it to be complete with respect to first order logic.





