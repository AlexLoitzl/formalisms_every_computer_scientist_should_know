\title{Formalisms Every Computer Scientist Should Know}
\author{Students in the class}

\newcommand{\booklicense}{
    This work is marked with CC0 1.0. 
    To view a copy of this license, visit http://creativecommons.org/publicdomain/zero/1.0
}

% Author subtitle
% For example, university or geographical location
\newcommand{\authorsubtitle}{ISTA}


\synctex=1

% PACKAGES

% this package allows large \fontsize
\usepackage{fix-cm}  
% this is for graphics. e.g. rectangle on title page
\usepackage{tikz}    
% Used by equations
\usepackage{amsmath} 
\usepackage{amssymb,amsfonts,amsthm,mathptmx}
% Paper dimensions
\usepackage[a4paper, total={6in, 8in}]{geometry}
% Inserting images
\usepackage{graphicx}
% Index creation
\usepackage{makeidx}
% Initialize an index so we can add entries with \index
\makeindex 
% Others
\usepackage{pbox}
\usepackage{stmaryrd}
% Links
\usepackage{hyperref}
% Convenient references
\usepackage{cleveref}

% MACROS


\newtheorem{theorem}{Theorem}
\newtheorem{definition}{Definition}
\newtheorem{example}{Example}
\newtheorem{homework}{Homework}

\newcommand{\N}{\mathbb{N}}
\newcommand{\R}{\mathbb{R}}
\renewcommand{\P}{\mathcal{P}}